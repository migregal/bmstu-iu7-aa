\chapter*{Заключение}
\addcontentsline{toc}{chapter}{Заключение}

В данной лабораторной работе были рассмотрены параллельные алгоритмы умножения матриц. 

Среди рассмотренных алгоритмов наиболе эффективным по времени яляется параллельный алгоритм умножения матриц по строкам, так как в нем отсутствуют лишние обращения к памяти. 

В связи с вышуказанным, параллельный алгоритм умножения по строкам является предпочтительным при обработке больших матриц в многопоточном окружении, однако, при работе с матрицами малых размерностей (меньше 64), стандартный алгоритм умножения становится более эффективным в связи с дополнительными затратами на организацию параллельности вычислений (создание потоков, организация совместного доступа к ресурсам).

В рамках выполнения работы решены следующие задачи.
\begin{itemize}
	\item изучtены основные методы параллельных вычислений;
	\item реализован каждый из указанных алгоритмов умножения матриц;
	\item проведено сравнение временных характеристик реализованных алгоритмов экспериментально.
\end{itemize}