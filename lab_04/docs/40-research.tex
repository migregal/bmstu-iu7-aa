\chapter{Экспериментальная часть}

В данном разделе будет проведено функциональное тестирование разработанного программного обеспечения. Так же будет произведено измерение временных характеристик каждого из реализованных алгоритмов. 

\section{Технические характеристики}

Технические характеристики устройства, на котором выполнялось исследование:

\begin{itemize}
	\item Процессор: Intel Core™ i5-8250U \cite{i5} CPU @ 1.60GHz.
	\item Память: 32 GiB.
	\item Операционная система: Manjaro \cite{manjaro} Linux \cite{linux} 21.1.4 64-bit.

\end{itemize}

Исследование проводилось на ноутбуке, включенном в сеть электропитания. Во время тестирования ноутбук был нагружен только встроенными приложениями окружения рабочего стола, окружением рабочего стола, а также непосредственно системой тестирования.

\section{Тестирование}

В данном разделе будет приведена таблица с тестами (таблица \ref{table:ref1}).

\begin{table}[h!]
	\begin{center}
	\caption{Таблица тестов}
	\label{table:ref1}
		\begin{tabular}{c@{\hspace{7mm}}c@{\hspace{7mm}}c@{\hspace{7mm}}c@{\hspace{7mm}}c@{\hspace{7mm}}c@{\hspace{7mm}}}
			\hline
			Первая матрица & Вторая матрица & Ожидаемый результат \\ \hline
			\vspace{4mm}
			$\begin{pmatrix}
			2
			\end{pmatrix}$ &
			$\begin{pmatrix}
			2
			\end{pmatrix}$ &
			$\begin{pmatrix}
			4
			\end{pmatrix}$ \\
			\vspace{2mm}
			\vspace{2mm}			
			$\begin{pmatrix}
			2 \\
			1 
			\end{pmatrix}$ &
			$\begin{pmatrix}
			1 & 2
			\end{pmatrix}$ &
			$\begin{pmatrix}
			2 & 4\\
			1 & 2
			\end{pmatrix}$\\
			\vspace{2mm}
			\vspace{2mm}
			$\begin{pmatrix}
			1 & 2 & 2\\
			1 & 2 & 2
			\end{pmatrix}$ &
			$\begin{pmatrix}
			1 & 2\\
			1 & 2\\
			1 & 2
			\end{pmatrix}$ &
			$\begin{pmatrix}
			5 & 10\\
			5 & 10
			\end{pmatrix}$ \\
			\vspace{2mm}
			\vspace{2mm}
			$\begin{pmatrix}
			1 & -2 & 3\\
			1 & 2 & 3\\
			1 & 2 & 3
			\end{pmatrix}$ &
			$\begin{pmatrix}
			-1 & 2 & 3\\
			1 & 2 & 3\\
			1 & 2 & 3
			\end{pmatrix}$ &
			$\begin{pmatrix}
			0 & 4 & 6\\
			4 & 12 & 18\\
			4 & 12 & 18
			\end{pmatrix}$\\
		\end{tabular}
	\end{center}
\end{table}
\newpage
При проведении функционального тестирования, полученные результаты работы программы совпали с ожидаемыми. Таким образом, функциональное тестирование пройдено успешно.

\section{Временные характеристики}

Для сравнения возьмем квадратные матрицы размерностью [32, 64, 128, \dots, 1024]. Количество потоков будем брать из набора [1, 2, 4, 8, \dots, $4*M$] где $M$ - количество логических ядер используемой ЭВМ.

Так как в общем случае вычисление произведения матриц является достаточно короткой задачей, воспользуемся усреднением массового эксперимента.  Для этого вычислим среднее арифметическое число тактов процессора, затраченных на выполнение алгоритма, для $n$ запусков. Сравнение произведем при $n = 100$.

Результаты замеров по результатам экспериментов для умножения по строкам приведены в Таблице \ref{tbl:time_rows}.

\begin{table}[ht]
	\small
	\begin{center}
		\caption{Замер времени для матриц, размером от 32 до 1024 элементов}
		\label{tbl:time_rows}
		\begin{tabular}{|c|c|c|c|c|c|c|}
			\hline
			\bfseries Размерность & \multicolumn{6}{c|}{\bfseries Количество потоков, ед} \\ \cline{2-7}
			\bfseries матрицы, эл. & \bfseries 1 & \bfseries 2 & \bfseries 4 & \bfseries 8 & \bfseries 16 & \bfseries 32
			\csvreader{inc/csv/time_sc.csv}{}
			{\\\hline \csvcoli&\csvcolii&\csvcoliv&\csvcolvi&\csvcolviii&\csvcolx&\csvcolxii}
			\\\hline
		\end{tabular}
	\end{center}
\end{table}

Результаты замеров по результатам экспериментов для умножения по столбцам приведены в Таблице \ref{tbl:time_cols}.

\begin{table}[ht]
	\small
	\begin{center}
		\caption{Замер времени для матриц, размером от 32 до 1024 элементов}
		\label{tbl:time_cols}
		\begin{tabular}{|c|c|c|c|c|c|c|}
			\hline
			\bfseries Размерность & \multicolumn{6}{c|}{\bfseries Количество потоков, ед} \\ \cline{2-7}
			\bfseries матрицы, эл. & \bfseries 1 & \bfseries 2 & \bfseries 4 & \bfseries 8 & \bfseries 16 & \bfseries 32
			\csvreader{inc/csv/time_sc.csv}{}
			{\\\hline \csvcoli&\csvcoliii&\csvcolv&\csvcolvii&\csvcolix&\csvcolxi&\csvcolxiii}
			\\\hline
		\end{tabular}
	\end{center}
\end{table}

На Рисунке \ref{plt:time_1024_cmp} отображены временные харкатеристики параллельных алгоритмов при размерностях квадратных матриц равных 1024 элемента.

\begin{figure}[ht]
	\centering
	\begin{tikzpicture}
		\begin{axis}[
			axis lines=left,
			xmin=1, xmax=32,
			xtick={1,2,4,8,16,32},
			xlabel={Размер массива, элементов},
			ylabel={Процессорное время, тиков},
			legend pos=north east,
			ymajorgrids=true
		]
			\addplot table[x=threads,y=rows,col sep=comma] {inc/csv/time_1024_sc.csv};
			\addplot table[x=threads,y=cols,col sep=comma] {inc/csv/time_1024_sc.csv};
			\legend{По строкам, По столбцам}
		\end{axis}
	\end{tikzpicture}
	\captionsetup{justification=centering}
	\caption{Временные характеристики на разном количестве потоков при матрицах размером 1024x1024}
	\label{plt:time_1024_cmp}
\end{figure}

Из Рисунка \ref{plt:time_1024_cmp} следует, что наиболее эффективными параллельные алгоритмы становятся при приближении числа потоков к количеству логических ядер используемой ЭВМ (8 для использованной в ходе эксперимента). На Рисунке \ref{plt:time_1024_long_cmp} приведены временные характеристики параллельных алгоритмов для числа потоков из набора [8, 16, 32].

\begin{figure}[ht]
	\centering
	\begin{tikzpicture}
		\begin{axis}[
			axis lines=left,
			xmin=1, xmax=32,
			xtick={1,2,4,8,16,32},
			xlabel={Размер массива, элементов},
			ylabel={Процессорное время, тиков},
			legend pos=north west,
			ymajorgrids=true
		]
			\addplot table[x=threads,y=rows,col sep=comma] {inc/csv/time_1024_long.csv};
			\addplot table[x=threads,y=cols,col sep=comma] {inc/csv/time_1024_long.csv};
			\legend{По строкам, По столбцам}
		\end{axis}
	\end{tikzpicture}
	\captionsetup{justification=centering}
	\caption{Временные характеристики на разном количестве потоков при матрицах размером 1024x1024}
	\label{plt:time_1024_long_cmp}
\end{figure}
\clearpage

Из Рисунка \ref{plt:time_1024_long_cmp} можно сделать вывод, что наиболее эффективны параллельные алгоритмы при количестве потоков, 16 (Удвоенное число логических ядердля использованной в ходе эксперимента ЭВМ). Объяснить это можно особенностью реализации многопоточности в конкретной системе (например, реализации \texttt{pthreads.h} \cite{pthread} через так называемые \texttt{Lightweight Threads (LWT)}).

В общем случае, наиболее эффективным будет являться конфигурация с числом потоков равным числу логических ядер используемов ЭВМ.

Сравним временные характеристики стандартного и параллельных алгоритмов при количестве потоков равном 8 (число логических ядер использованной ЭВМ) при разных размерностях квадратных матриц. Данные для сравнения получены из Таблиц \ref{tbl:time_rows} и \ref{tbl:time_cols}.

\begin{figure}[ht]
	\centering
	\begin{tikzpicture}
		\begin{axis}[
			axis lines=left,
			xmin=32, xmax=1024,
			xtick={32, 128, 256, 512, 1024},
			xlabel={Размер массива, элементов},
			ylabel={Процессорное время, тиков},
			legend pos=north west,
			ymajorgrids=true
		]
			\addplot table[x=len,y=base,col sep=comma] {inc/csv/time_opt.csv};
			\addplot table[x=len,y=rows,col sep=comma] {inc/csv/time_opt.csv};
			\addplot table[x=len,y=cols,col sep=comma] {inc/csv/time_opt.csv};
			\legend{Стандартный, По строкам, По столбцам}
		\end{axis}
	\end{tikzpicture}
	\captionsetup{justification=centering}
	\caption{Временные характеристики реализованных алгоритмов при количестве потоков равном 8}
	\label{plt:time_cmp}
\end{figure}

Как и ожидалось, параллельные алгоритмы оказались более эффективными по времени, чем стандартный алгоритм. При этом, параллеьльный алгоритм по строкам является наиболее эффективным из рассмотренных. Это связано с аппаратными особенностями работы с памятью - при умножении по столбцам на каждой итерации 2 цикла происходит обращение к новой области памяти, что приводит к большим временным затратам. В алгоритме умножения по строкам область памяти меняется только в первом цикле, поэтому он явлется более эффективным.

Однако, при работе с матрицами малых размерностей (менее 64 элементов) стандартный алгоритм оказался более эффективным (в 4 раза в сравнении с 32 поточной параллельной реализацией) по времени, так как параллельные реализации требуют дополнительных затрат по времени и памятидля  реализацию многопоточности (создание потоков, реализация совместного доступа к ресурсам).

\section{Вывод}
В данном разделе было произведено сравнение количества затраченного вре­мени вышеизложенных алгоритмов.

Наиболее эффективной по времени при работе с матрицами больших размерностей (более 64 элементов) оказалась параллельная реализация на 16 потоках, что может быть обусловлено особенностью конкретной реализации библиотеки \texttt{pthreads} \cite{pthread} для данной операционной системы. В общем случае, наиболее эффективной является реализация с числом потоков равным числу логических ядер используемой ЭВМ.

При работе с матрицами малых размерностей (менее 64 элементов) стандартный алгоритм оказался более эффективным (в 4 раза в сравнении с 32 поточной параллельной реализацией) по времени, что связано с дополнительным затратами на реализацию многопоточности (создание потоков, реализация совместного доступа к ресурсам).