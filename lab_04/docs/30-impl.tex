\chapter{Технологическая часть}

\section{Выбор средств реализации}

В качестве языка программирования для реализации данной лабораторной работы был выбран язык C \cite{c_lang}. Данный выбор обусловлен тем, что я имею некоторый опыт разработки на нем, а так же поддержкой данным языком нативных потоков посредством использования библиотеки pthreads \cite{pthread}.

\section{Требования к программному обеспечению}

Входными данными являются две матрицы $A$ и $B$.
Количество столбцов матрицы A должно быть равно количеству строк матрицы $B$. 

На выходе получается результат умножения введенных пользователем матриц.

\section{Сведения о модулях программы}

Данная программа разбита на модули:
\begin{itemize}
    \item \texttt{main.c} - файл, содержащий точку входа в программу;
    \item \texttt{matrix.h} - файл, содержащий определение структуры матрицы;
    \item \texttt{matrix.c} - файл, содержащий реализации основных функций для работы с матрицами;
    \item \texttt{process.c} - файл содержащий логику работы приложения. В этом файле происходит общение с пользователем и вызов алгоритмов;
    \item \texttt{parallel\_process.c} - файл содержащий логику работы в параллельном режиме. В этом файле происходят замеры временных характеристик для алгоритмов с разными входными данными и количеством потоков;
    \item \texttt{multiplication.c} - файл, содержащий реализацию простого алгоритма умножения матриц;
    \item \texttt{parallel\_multiplication.c} - файл, содержащий реализации параллельных алгоритмов умножения матриц;
    \item \texttt{threads.h} - файл, содержащий определение структуры аргументов, передаваемых функциям реализующим параллельные алгоритмы;
    \item \texttt{threads.c} - файл, содержащий реализацию функции распараллеливания вычислений;
    \item \texttt{timer.c} - файл, содержащий реализацию функции вычисления текущего количества тиков;
    \item \texttt{matrix\_io.c} - файл, содержащий реализации различных функций ввода-вывода матриц;
    \item \texttt{io.c} - файл, содержащий реализации различных функций ввода-вывода;
\end{itemize}

На листингах \ref{lst:main} -- \ref{lst:matrix_io} представлен код программы.

\listingfile{main.c}{main}{C}{Основной файл программы main}{}

\listingfile{matrix.h}{matrix_h}{C}{Определение структур и методов для работы с матрицами matrix.h}{linerange={10-38}}

\clearpage

\listingfile{matrix.c}{matrix_с}{C}{Реализация методов для работы с матрицами matrix.c}{linerange={9-106}}

\clearpage

\listingfile{process.c}{process_с}{C}{Реализация интерактивного процесса process.c}{linerange={11-48}}

\clearpage

\listingfile{parallel_process.c}{parallel_process}{C}{Реализация замеров временных характеристик для разных алгоритмов parallel\_process.c}{linerange={11-65}}

\clearpage

\listingfile{threads.h}{threads_h}{C}{Определения структур аргументов для параллельных реализаций алгоритмов threads.h}{}

\clearpage

\listingfile{threads.c}{threads_c}{C}{Реализация распараллеливания алгоритмов thread.c}{linerange={8-50}}

\clearpage

\listingfile{multiplication.c}{multiplication}{C}{Реализация стандартного алгоритма умножения}{linerange={5-38}}

\listingfile{timer.c}{timer}{C}{Реализация функции вычисления текущего просеррного времени timer.c}{linerange={6-8}}

\clearpage

\listingfile{parallel_multiplication.c}{parallel_multiplication}{C}{Реализация параллельных алгоритмов умножения}{linerange={8-44}}

\clearpage

\listingfile{io.c}{io}{C}{Реализация функций ввода-вывода io.c}{linerange={7-50}}

\clearpage

\listingfile{matrix_io.c}{matrix_io}{C}{Реализация функций ввода-вывода для матриц matrix\_io.c}{linerange={8-58}}

\clearpage

\section{Вывод}

В данном разделе были реализованы вышеописанные алгоритмы. 

Было разработано программное обеспечение, удовлетворяющее предъявляемым требованиям. Так же были представлены соответствующие листинги \ref{lst:main} -- \ref{lst:matrix_io} с кодом программы.