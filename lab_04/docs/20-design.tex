\chapter{Конструкторская часть}

В данном разделе будут рассмотрены схемы алгоритмов умножения матриц и модель вычислений.

\section{Схемы}
На рисунке \ref{img:std_mul} представлена схема стандартного алгоритма умножения матриц.

\imgw{std_mul}{ht!}{\textwidth}{Схема стандартного алгоритма умножения матриц}
\clearpage

На рисунках \ref{img:parallel_rows} -- \ref{img:parallel_cols} представлены схемы параллельных алгоритмов умножения матриц.

\imgw{parallel_rows}{ht!}{\textwidth}{Схема параллельного алгоритма по строкам}

\clearpage

\imgw{parallel_cols}{ht!}{\textwidth}{Схема параллельного алгоритма по столбцам}

\clearpage

\imgw{thread_start}{ht!}{\textwidth}{Схема функции создания потоков}

\clearpage

\imgw{parallel_loop_1}{ht!}{\textwidth}{Схема с параллельным выполнением первого цикла}

\clearpage

\imgw{parallel_loop_2}{ht!}{\textwidth}{Схема с параллельным выполнением второго цикла}

\section{Вывод}

На основе теоретических данных, полученных из аналитического раздела, была построена схема стандартного алгоритма умножения матриц (рис. \ref{img:std_mul}),а так же были построены схемы двух вариантов параллельного выполнения данного алгоритма (рис. \ref{img:thread_start} -- \ref{img:parallel_rows}).