\chapter{Конструкторская часть}

В данном разделе будут рассмотрены схемы вышеизложенных алгоритмов. Так же будут описаны структуры данных, приведены структура программного обеспечения и классы эквивалентности для тестирования реализуемого программного обеспечения.

\section{Схемы}

На рисунке \ref{img:brute} представлена схема наивного алгоритма решения задачи коммивояжера.

\imgw{brute}{ht!}{0.5\textwidth}{Схема наивного алгоритма}
\clearpage

На рисунке \ref{img:ant} представлена схема наивного алгоритма решения задачи коммивояжера.

\imgw{ant}{ht!}{0.3\textwidth}{Схема муравьиного алгоритма}

\section{Описание структуры программного обеспечения}

На Рисунке \ref{img:uml} представлена диаграмма классов реализуемого программного обеспечения.

\imgw{uml}{ht!}{0.5\textwidth}{Схема муравьиного алгоритма}

В связи с тем, что в жадном алгоритме не используются дополнительные параметры, структуры необходимы лишь для реализации муравьиного алгоритма.

\section{Описание структур данных}

Для реализации муравьиного алгоритма введем некоторые пользовательские типы данных:

\begin{itemize}
    \item AntColony -- структура, описывающая колонию муравьев;
    \item Ant -- структура, описывающая конкретного муравьяв колонии.
\end{itemize}

\subsection{Описание пользовательского типа данных AntColony}

\listingfile{types.go}{antcolony}{Go}{Описание пользовательского типа данных AntColony}{linerange={3-7}}

Здесь:
\begin{itemize}
    \item \texttt{w} -- рассматриваемый граф, представленный матрицей смежности;
    \item \texttt{ph} -- матрица, элемент $(i, j)$ которой соответствует количеству феромона на участке дороги между городами $i$ и $j$;
    \item \texttt{a} -- значение параметра $\alpha$;
    \item \texttt{b} -- значение параметра $\beta$;
    \item \texttt{q} -- значение параметра $Q$;
    \item \texttt{p} -- значение параметра $\rho$.
\end{itemize}

\subsection{Описание пользовательского типа данных Ant}

\listingfile{types.go}{antdef}{Go}{Описание пользовательского типа данных Ant}{linerange={9-15}}

Здесь:
\begin{itemize}
    \item \texttt{col} -- указатель на колонию, которой ринадлежит данный муравей;
    \item \texttt{vis} -- рассматриваемый граф, представленный матрицей смежности;
    \item \texttt{isv} -- рассматриваемый граф, представленный матрицей признаков посещения (элемент $(i, j$ данной матрицы показывает, был ли посещен город $j$ из города $i$ данным муравьем);
    \item \texttt{pos} -- номер города, в котором находится муравей.
\end{itemize}

\section{Тестирование}

В рамках данной лабораторной работы можно выделить следующие классы эквивалентности:

\begin{itemize}
    \item входными данными является ациклический ориентированный взвешенный граф;
    \item входными данными является циклический ориентированный взвешенный граф.
\end{itemize}

Для проверки корректности работы реализованных алгоритмов будет проводится функциональное тестирование согласно классам эквивалентности.

\section{Вывод}

На основе теоретических данных, полученных из аналитического раздела,
были построены схемы (рисунки \ref{img:brute} -- \ref{img:ant}) двух алгоритмов решения задачи коммивояжера.