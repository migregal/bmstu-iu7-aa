\chapter{Технологическая часть}

В данном разделе приведены требования к программному обеспечению, средства реализации и листинги кода.

\section{Выбор средств реализации}

В качестве языка программирования для реализации данной лабораторной работы был выбран язык Golang \cite{golang}. Данный выбор обусловлен тем, что я имею некоторый опыт разработки на нем, а так же наличием у языка встроенных высокоточных средств тестирования и анализа разработанного ПО. 

\section{Требования к программному обеспечению}

Входными данными являются:
\begin{itemize}
    \item граф, представленный матрицей смежности;
\end{itemize}

На выходе - решение задачи коммивояжера для введенного графа.


\section{Сведения о модулях программы}

Данная программа разбита на следующие модули:
\begin{itemize}
\item \texttt{main.go} -- Файл, содержащий точку входа в программу.
\item \texttt{compare.go} -- Файл содержит основную логику приложения.
\item \texttt{utils.go} -- Файл содержит реализацию утилитарных функций.
\item \texttt{io.go} -- Файл содержит реализацию функции ввода-вывода.
\item \texttt{types.go} -- Файл содержит определения и свпомогательные функции для пользовательских типов.
\item \texttt{brute.go} -- Файл содержит реализацию жадного алгоритма.
\item \texttt{ant.go} -- Файл содержит реализацию муравьиного алгоритма.
\end{itemize}

В листингах \ref{lst:main}--\ref{lst:ant_4} представлены исходные коды разобранных ранее алгоритмов.
\newpage

\listingfile{main.go}{main}{Go}{Основной файл программы main}{}

\listingfile{compare.go}{compare}{Go}{Основная логика приложения}{linerange={8-29}}

\listingfile{utils.go}{utils}{Go}{Утилитарные функции}{linerange={10-30}}

\listingfile{io.go}{io}{Go}{Функции ввода-вывода}{linerange={13-45}}

\clearpage

\listingfile{types.go}{types}{Go}{Определения пользовательских типов данных}{linerange={3-56}}

\clearpage

\listingfile{brute.go}{brute_1}{Go}{Реализация жадного алгоритма. Часть 1}{linerange={5-50}}

\clearpage

\listingfile{brute.go}{brute_2}{Go}{Реализация жадного алгоритма. Часть 2}{linerange={51-59}}

\listingfile{ant.go}{ant_1}{Go}{Реализация муравьиного алгоритма. Часть 1}{linerange={9-37}}

\clearpage

\listingfile{ant.go}{ant_2}{Go}{Реализация муравьиного алгоритма. Часть 2}{linerange={38-74}}

\clearpage

\listingfile{ant.go}{ant_3}{Go}{Реализация муравьиного алгоритма. Часть 3}{linerange={75-115}}

\clearpage

\listingfile{ant.go}{ant_4}{Go}{Реализация муравьиного алгоритма. Часть 4}{linerange={116-125}}

\section{Тестирование}

В таблице \ref{tbl:functional_test} приведены функциональные тесты для алгоритмов сортировки.

\begin{table}[h!]
	\begin{center}
	\caption{Таблица тестов}
	\label{tbl:functional_test}
		\begin{tabular}{c@{\hspace{7mm}}c@{\hspace{7mm}}c@{\hspace{7mm}}c@{\hspace{7mm}}c@{\hspace{7mm}}c@{\hspace{7mm}}c@{\hspace{7mm}}}
			\hline
			Первая матрица & Ожидаемый результат & Жадный & Муравьиный\\ \hline
			\vspace{4mm}
			$\begin{bmatrix}
			0 & 3 & 1 & 6 & 8\\
			3 & 0 & 4 & 1 & 0\\
			1 & 4 & 0 & 5 & 0\\
			6 & 1 & 5 & 6 & 1\\
			8 & 0 & 0 & 1 & 1
			\end{bmatrix}$ &
			15 &
			15 &
			15 \\
			\vspace{2mm}
			\vspace{2mm}
			$\begin{bmatrix}
			0 & 10 & 15 & 20\\
			10 & 0 & 35 & 25 \\
			15 & 35 & 0 & 30 \\
			20 & 25 & 30 & 0
			\end{bmatrix}$ &
			80 &
			80 &
			80 \\
		\end{tabular}
	\end{center}
\end{table}

При проведении функционального тестирования, полученные результаты работы программы совпали с ожидаемыми. Таким образом, функциональное тестирование пройдено успешно.

\section{Вывод}

Были реализованы спроектированные алгоритмы: жадный алгоритм и муравьиный алгоритм.