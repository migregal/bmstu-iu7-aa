\chapter*{Введение}
\addcontentsline{toc}{chapter}{Введение}

Задача коммивояжёра – задача транспортной логистики, отрасли, занимающейся планированием транспортных перевозок. Коммивояжёру, чтобы распродать нужные и не очень нужные в хозяйстве товары, следует объехать $n$ пунктов и в конце концов вернуться в исходный пункт. Требуется определить наиболее выгодный маршрут объезда. В качестве меры выгодности маршрута может служить суммарное время в пути, суммарная стоимость дороги, или, в простейшем случае, длина маршрута.

Муравьиный алгоритм – один из эффективных полиномиальных алгоритмов для нахождения приближённых решений задачи коммивояжёра, а также решения аналогичных задач поиска маршрутов на графах. 

Суть подхода заключается в анализе и использовании модели поведения муравьёв, ищущих пути от колонии к источнику питания, и представляет собой метаэвристическую оптимизацию.

Целью данной работы является реализация и изучение следующих алгоритмов решения задачи коммивояжера:
\begin{itemize}
	\item муравьиный алгоритм;
	\item наивный алгоритм.
\end{itemize}

Для достижения данной цели необходимо решить следующие задачи:

\begin{itemize}
    \item изучить подходы к решению задачи коммивояжера;
    \item привести схемы рассматриваемых алгоритмов, а именно:
	\begin{itemize}
	    \item жадный алгоритм решения задачи коммивояжера;
	    \item муравьиный алгоритм решения задачи коммивояжера;
	\end{itemize}
	\item описать использующиеся структуры данных;
	\item описать структуру разрабатываемого програмного обеспечения;
	\item определить средства программной реализации;
	\item определить требования к программному обеспечению;
	\item привести сведения о модулях программы;
	\item провести тестирование реализованного программного обеспечения;
	\item провести экспериментальные замеры временных характеристик реализованных алгоритмов.
\end{itemize}
