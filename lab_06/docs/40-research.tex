\chapter{Экспериментальная часть}

В данном разделе будет проведено функциональное тестирование разработанного программного обеспечения. Так же будет произведено измерение временных характеристик каждого из реализованных алгоритмов. 

Для проведения подобных экспериментов на языке программирования \texttt{Golang} \cite{golang}, используется специальный пакет \texttt{time} \cite{gotime}, позволяющий замерить процессорное время при помощи функции \texttt{time.Time}.

\section{Технические характеристики}

Технические характеристики устройства, на котором выполнялось исследование:

\begin{itemize}
	\item процессор: Intel Core™ i5-8250U \cite{i5} CPU @ 1.60GHz;
	\item память: 32 GiB;
	\item Операционная система: Manjaro \cite{manjaro} Linux \cite{linux} 21.1.4 64-bit.
\end{itemize}

Исследование проводилось на ноутбуке, включенном в сеть электропитания. Во время тестирования ноутбук был нагружен только встроенными приложениями окружения рабочего стола, окружением рабочего стола, а также непосредственно системой тестирования.

\section{Временные характеристики}

Для сравнения графы размерностью [2, 3, 4, \dots, 10]. 
Результаты замеров по результатам экспериментов приведены в Таблице \ref{tbl:time}.

\begin{table}[ht]
	\small
	\begin{center}
		\caption{Замер времени для графов размером от 2 до 10 узлов}
		\label{tbl:time}
		\begin{tabular}{|c|c|c|}
			\hline
			& \multicolumn{2}{c|}{\bfseries Время, с} \\ \cline{1-3}
			\bfseries Размерность графа, эл.& \bfseries Наивный & \bfseries Муравьиный
			\csvreader{inc/csv/time.csv}{}
			{\\\hline \csvcoli&\csvcolii&\csvcoliii}
			\\\hline
		\end{tabular}
	\end{center}
\end{table}

На рисунке \ref{plt:time} приведены результаты сравнения на основе данных, представленных в Таблице \ref{tbl:time}.

\begin{figure}[ht]
	\centering
	\begin{tikzpicture}
		\begin{axis}[
			axis lines=left,
			xlabel={Длина массива, эл},
			ylabel={Время, с},
			legend pos=north west,
			ymajorgrids=true
		]
			\addplot table[x=size,y=brute,col sep=comma] {inc/csv/time.csv};
			\addplot table[x=size,y=ant,col sep=comma] {inc/csv/time.csv};
			\legend{Жадный,Муравьиный}
		\end{axis}
	\end{tikzpicture}
	\captionsetup{justification=centering}
	\caption{Сравнение времени работы алгоритмов}
	\label{plt:time}
\end{figure}

Из данных, приведенных в  Таблице \ref{tbl:time}, видно, что муравьиный алгоритм становится эффективнее жадного алгоритма при размере графа, не меньше 9 элементов. В связи с этим, при больших размерностях графа муравьиный алгоритм является более предпочтительным для использования.

\section{Автоматическая параметризация}

Рассмотрим результаты автоматической пармметризации на двух различных классах данных.

\subsection{Класс данных 1}
\begin{equation}
    \label{matrix}
    M = \begin{pmatrix}
        0 &    21 &    21 &     4 &    22 &    20 &     6 &     4 &    11 &    23 \\
        21 &    0 &    22 &     6 &     9 &    21 &    16 &     6 &     2 &     7 \\
        21 &   22 &     0 &    18 &    23 &     7 &    19 &    19 &     1 &    14 \\
        4 &     6 &    18 &     0 &     7 &    11 &    16 &    10 &     8 &    19 \\
        22 &    9 &    23 &     7 &     0 &    15 &     2 &    23 &     1 &    22 \\
        20 &   21 &     7 &    11 &    15 &     0 &     6 &     7 &     8 &    21 \\
        6 &    16 &    19 &    16 &     2 &     6 &     0 &     8 &     3 &    14 \\
        4 &     6 &    19 &    10 &    23 &     7 &     8 &     0 &    23 &    15 \\
        11 &    2 &     1 &     8 &     1 &     8 &     3 &    23 &     0 &     9 \\
        23 &    7 &    14 &    19 &    22 &    21 &    14 &    15 &     9 &     0 \\
    \end{pmatrix}
\end{equation}

В таблице ~\ref{tab:log1} приведены результаты параметризации метода решения задачи коммивояжера на основании муравьиного алгоритма. Количество дней было взято равным 50. Полный перебор определил оптимальную длину пути 52. Два последних стобца таблицы определяют найденный муравьиным алгоритм оптимальный путь и разницу этого пути с оптимальным путём, найденным алгоритмом полного перебора.

\begin{table}[ht]
    \caption{Таблица коэффициентов для класса данных 1.}
    \begin{minipage}[h!]{0.10\hsize}\centering
        \begin{center}\resizebox{4\textwidth}{!}{
                \begin{tabular}{|c@{\hspace{5mm}}|c@{\hspace{5mm}}|c@{\hspace{5mm}}|c@{\hspace{5mm}}|c@{\hspace{5mm}}|c|}
                    \hline
                    $\alpha$ & $\beta$ & $\rho$ & Длина & Разница \\
                    \hline
                    0    & 1    & 0    & 52    & 0     \\
                    0    & 1    & 0.1  & 52    & 0     \\
                    0    & 1    & 0.2  & 52    & 0     \\
                    0    & 1    & 0.3  & 52    & 0     \\
                    0    & 1    & 0.4  & 53    & 1     \\
                    0    & 1    & 0.5  & 52    & 0     \\
                    0    & 1    & 0.6  & 52    & 0     \\
                    0    & 1    & 0.7  & 52    & 0     \\
                    0    & 1    & 0.8  & 52    & 0     \\
                    0    & 1    & 0.9  & 52    & 0     \\
                    0    & 1    & 1    & 52    & 0     \\
                    0.1  & 0.9  & 0    & 52    & 0     \\
                    0.1  & 0.9  & 0.1  & 52    & 0     \\
                    0.1  & 0.9  & 0.2  & 53    & 1     \\
                    0.1  & 0.9  & 0.3  & 52    & 0     \\
                    0.1  & 0.9  & 0.4  & 52    & 0     \\
                    0.1  & 0.9  & 0.5  & 52    & 0     \\
                    0.1  & 0.9  & 0.6  & 53    & 1     \\
                    0.1  & 0.9  & 0.7  & 52    & 0     \\
                    0.1  & 0.9  & 0.8  & 52    & 0     \\
                    0.1  & 0.9  & 0.9  & 52    & 0     \\
                    0.1  & 0.9  & 1    & 52    & 0     \\
                    0.2  & 0.8  & 0    & 52    & 0     \\
                    0.2  & 0.8  & 0.1  & 52    & 0     \\
                    0.2  & 0.8  & 0.2  & 52    & 0     \\
                    0.2  & 0.8  & 0.3  & 53    & 1     \\
                    0.2  & 0.8  & 0.4  & 52    & 0     \\
                    0.2  & 0.8  & 0.5  & 52    & 0     \\
                    0.2  & 0.8  & 0.6  & 53    & 1     \\
                    0.2  & 0.8  & 0.7  & 53    & 1     \\
                    0.2  & 0.8  & 0.8  & 52    & 0     \\
                    0.2  & 0.8  & 0.9  & 52    & 0     \\
                    0.2  & 0.8  & 1    & 52    & 0     \\
                    0.3  & 0.7  & 0    & 53    & 1     \\
                    0.3  & 0.7  & 0.1  & 52    & 0     \\
                    0.3  & 0.7  & 0.2  & 52    & 0     \\
                    0.3  & 0.7  & 0.3  & 53    & 1     \\
                    0.3  & 0.7  & 0.4  & 52    & 0     \\
                    0.3  & 0.7  & 0.5  & 52    & 0     \\
                    0.3  & 0.7  & 0.6  & 52    & 0     \\
                    \hline
                \end{tabular}}
                \label{tab:log1}
        \end{center}
    \end{minipage}
    \hfill
    \begin{minipage}[!h]{0.50\hsize}\centering
        \begin{center}\resizebox{0.8\textwidth}{!}{%
                \begin{tabular}{|c@{\hspace{5mm}}|c@{\hspace{5mm}}|c@{\hspace{5mm}}|c@{\hspace{5mm}}|c@{\hspace{5mm}}|c|}
                    \hline
                    $\alpha$ & $\beta$ & $\rho$ & Длина & Разница \\
                    \hline
                    0.3  & 0.7  & 0.7  & 52    & 0     \\
                    0.3  & 0.7  & 0.8  & 52    & 0     \\
                    0.3  & 0.7  & 0.9  & 53    & 1     \\
                    0.3  & 0.7  & 1    & 52    & 0     \\
                    0.4  & 0.6  & 0    & 53    & 1     \\
                    0.4  & 0.6  & 0.1  & 53    & 1     \\
                    0.4  & 0.6  & 0.2  & 53    & 1     \\
                    0.4  & 0.6  & 0.3  & 52    & 0     \\
                    0.4  & 0.6  & 0.4  & 53    & 1     \\
                    0.4  & 0.6  & 0.5  & 52    & 0     \\
                    0.4  & 0.6  & 0.6  & 52    & 0     \\
                    0.4  & 0.6  & 0.7  & 52    & 0     \\
                    0.4  & 0.6  & 0.8  & 52    & 0     \\
                    0.4  & 0.6  & 0.9  & 52    & 0     \\
                    0.4  & 0.6  & 1    & 52    & 0     \\
                    0.5  & 0.5  & 0    & 52    & 0     \\
                    0.5  & 0.5  & 0.1  & 52    & 0     \\
                    0.5  & 0.5  & 0.2  & 52    & 0     \\
                    0.5  & 0.5  & 0.3  & 53    & 1     \\
                    0.5  & 0.5  & 0.4  & 52    & 0     \\
                    0.5  & 0.5  & 0.5  & 52    & 0     \\
                    0.5  & 0.5  & 0.6  & 52    & 0     \\
                    0.5  & 0.5  & 0.7  & 52    & 0     \\
                    0.5  & 0.5  & 0.8  & 52    & 0     \\
                    0.5  & 0.5  & 0.9  & 52    & 0     \\
                    0.5  & 0.5  & 1    & 52    & 0     \\
                    0.6  & 0.4  & 0    & 53    & 1     \\
                    0.6  & 0.4  & 0.1  & 53    & 1     \\
                    0.6  & 0.4  & 0.2  & 56    & 4     \\
                    0.6  & 0.4  & 0.3  & 52    & 0     \\
                    0.6  & 0.4  & 0.4  & 52    & 0     \\
                    0.6  & 0.4  & 0.5  & 55    & 3     \\
                    0.6  & 0.4  & 0.6  & 56    & 4     \\
                    0.6  & 0.4  & 0.7  & 52    & 0     \\
                    0.6  & 0.4  & 0.8  & 53    & 1     \\
                    0.6  & 0.4  & 0.9  & 53    & 1     \\
                    0.6  & 0.4  & 1    & 52    & 0     \\
                    0.7  & 0.3  & 0    & 52    & 0     \\
                    0.7  & 0.3  & 0.1  & 53    & 1     \\
                    0.7  & 0.3  & 0.2  & 52    & 0     \\
                    \hline
                \end{tabular}}
        \end{center}
    \end{minipage}
\end{table}
\clearpage
\begin{table}[!h]
    \begin{center}
        \begin{tabular}{|c@{\hspace{7mm}}|c@{\hspace{7mm}}|c@{\hspace{7mm}}|c@{\hspace{7mm}}|c@{\hspace{7mm}}|c|}
            \hline
            $\alpha$ & $\beta$ & $\rho$ & Длина & Разница \\
            \hline
            0.7  & 0.3  & 0.3  & 52    & 0     \\
            0.7  & 0.3  & 0.4  & 53    & 1     \\
            0.7  & 0.3  & 0.5  & 53    & 1     \\
            0.7  & 0.3  & 0.6  & 52    & 0     \\
            0.7  & 0.3  & 0.7  & 53    & 1     \\
            0.7  & 0.3  & 0.8  & 57    & 5     \\
            0.7  & 0.3  & 0.9  & 52    & 0     \\
            0.7  & 0.3  & 1    & 52    & 0     \\
            0.8  & 0.2  & 0    & 59    & 7     \\
            0.8  & 0.2  & 0.1  & 53    & 1     \\
            0.8  & 0.2  & 0.2  & 56    & 4     \\
            0.8  & 0.2  & 0.3  & 53    & 1     \\
            0.8  & 0.2  & 0.4  & 52    & 0     \\
            0.8  & 0.2  & 0.5  & 56    & 4     \\
            0.8  & 0.2  & 0.6  & 53    & 1     \\
            0.8  & 0.2  & 0.7  & 52    & 0     \\
            0.8  & 0.2  & 0.8  & 52    & 0     \\
            0.8  & 0.2  & 0.9  & 52    & 0     \\
            0.8  & 0.2  & 1    & 53    & 1     \\
            0.9  & 0.1  & 0    & 56    & 4     \\
            0.9  & 0.1  & 0.1  & 53    & 1     \\
            0.9  & 0.1  & 0.2  & 52    & 0     \\
            0.9  & 0.1  & 0.3  & 56    & 4     \\
            0.9  & 0.1  & 0.4  & 52    & 0     \\
            0.9  & 0.1  & 0.5  & 53    & 1     \\
            0.9  & 0.1  & 0.6  & 56    & 4     \\
            0.9  & 0.1  & 0.7  & 56    & 4     \\
            0.9  & 0.1  & 0.8  & 55    & 3     \\
            0.9  & 0.1  & 0.9  & 53    & 1     \\
            0.9  & 0.1  & 1    & 53    & 1     \\
            1    & 0    & 0    & 71    & 19    \\
            1    & 0    & 0.1  & 61    & 9     \\
            1    & 0    & 0.2  & 53    & 1     \\
            1    & 0    & 0.3  & 59    & 7     \\
            1    & 0    & 0.4  & 59    & 7     \\
            1    & 0    & 0.5  & 60    & 8     \\
            1    & 0    & 0.6  & 60    & 8     \\
            1    & 0    & 0.7  & 74    & 22    \\
            1    & 0    & 0.8  & 60    & 8     \\
            1    & 0    & 0.9  & 57    & 5     \\
            1    & 0    & 1    & 60    & 8     \\
            \hline
        \end{tabular}
    \end{center}
\end{table}
\clearpage

\subsection{Класс данных 2}

\begin{equation}
    \label{matrix1}
    M = \begin{pmatrix}
        0 &  1790 &   200 &  1900 &    63 &  1659 &  1820 &  1395 &  2382 &   649 \\
        1790 &     0 &  1573 &  2435 &  1515 &   714 &   892 &  2193 &  1590 &  1003 \\
        200 &  1573 &     0 &   833 &   392 &  2404 &   962 &   902 &   141 &  1123 \\
        1900 &  2435 &   833 &     0 &  2283 &  1652 &  2362 &  2262 &  1512 &  2166 \\
        63 &  1515 &   392 &  2283 &     0 &  1322 &   290 &  1305 &  2100 &   969 \\
        1659 &   714 &  2404 &  1652 &  1322 &     0 &   256 &    78 &  2236 &  2041 \\
        1820 &   892 &   962 &  2362 &   290 &   256 &     0 &  1180 &  1547 &  1279 \\
        1395 &  2193 &   902 &  2262 &  1305 &    78 &  1180 &     0 &  1640 &  1161 \\
        2382 &  1590 &   141 &  1512 &  2100 &  2236 &  1547 &  1640 &     0 &  2212 \\
        649 &  1003 &  1123 &  2166 &   969 &  2041 &  1279 &  1161 &  2212 &     0 \\
    \end{pmatrix}
\end{equation}


В таблице ~\ref{tab:log2} приведены результаты параметризации метода решения задачи коммивояжера на основании муравьиного алгоритма для матрицы с элементами в диапазоне $[0, 2500]$. Количество дней было взято равным 50. Полный перебор определил оптимальную длину пути 6986. Два последних стобца таблицы определяют найденный муравьиным алгоритм оптимальный путь и разницу этого пути с оптимальным путём, найденным алгоритмом полного перебора.

\begin{table}
    \caption{Таблица коэффициентов для класса данных №2}
    \begin{minipage}[h!]{0.10\hsize}\centering
        \begin{center}\resizebox{4\textwidth}{!}{%
                \begin{tabular}{|c@{\hspace{5mm}}|c@{\hspace{5mm}}|c@{\hspace{5mm}}|c@{\hspace{5mm}}|c@{\hspace{5mm}}|c|}
                    \hline
                    $\alpha$ & $\beta$ & $\rho$ & Длина & Разница \\
                    \hline
                    0    & 1    & 0    & 6986  & 0     \\
                    0    & 1    & 0.1  & 6986  & 0     \\
                    0    & 1    & 0.2  & 6986  & 0     \\
                    0    & 1    & 0.3  & 6986  & 0     \\
                    0    & 1    & 0.4  & 6986  & 0     \\
                    0    & 1    & 0.5  & 6986  & 0     \\
                    0    & 1    & 0.6  & 6986  & 0     \\
                    0    & 1    & 0.7  & 6986  & 0     \\
                    0    & 1    & 0.8  & 6986  & 0     \\
                    0    & 1    & 0.9  & 6992  & 6     \\
                    0    & 1    & 1    & 6986  & 0     \\
                    0.1  & 0.9  & 0    & 6986  & 0     \\
                    0.1  & 0.9  & 0.1  & 6992  & 6     \\
                    0.1  & 0.9  & 0.2  & 6986  & 0     \\
                    0.1  & 0.9  & 0.3  & 6986  & 0     \\
                    0.1  & 0.9  & 0.4  & 6986  & 0     \\
                    0.1  & 0.9  & 0.5  & 6986  & 0     \\
                    0.1  & 0.9  & 0.6  & 6986  & 0     \\
                    0.1  & 0.9  & 0.7  & 6986  & 0     \\
                    0.1  & 0.9  & 0.8  & 6986  & 0     \\
                    0.1  & 0.9  & 0.9  & 7165  & 179   \\
                    0.1  & 0.9  & 1    & 6986  & 0     \\
                    0.2  & 0.8  & 0    & 6986  & 0     \\
                    0.2  & 0.8  & 0.1  & 6986  & 0     \\
                    0.2  & 0.8  & 0.2  & 6986  & 0     \\
                    0.2  & 0.8  & 0.3  & 6992  & 6     \\
                    0.2  & 0.8  & 0.4  & 6992  & 6     \\
                    0.2  & 0.8  & 0.5  & 6992  & 6     \\
                    0.2  & 0.8  & 0.6  & 6986  & 0     \\
                    0.2  & 0.8  & 0.7  & 6992  & 6     \\
                    0.2  & 0.8  & 0.8  & 6986  & 0     \\
                    0.2  & 0.8  & 0.9  & 6986  & 0     \\
                    0.2  & 0.8  & 1    & 6986  & 0     \\
                    0.3  & 0.7  & 0    & 6986  & 0     \\
                    0.3  & 0.7  & 0.1  & 6986  & 0     \\
                    0.3  & 0.7  & 0.2  & 7139  & 153   \\
                    0.3  & 0.7  & 0.3  & 7139  & 153   \\
                    0.3  & 0.7  & 0.4  & 6986  & 0     \\
                    0.3  & 0.7  & 0.5  & 6986  & 0     \\
                    0.3  & 0.7  & 0.6  & 6986  & 0     \\
                    \hline
                \end{tabular}}
                \label{tab:log2}
        \end{center}
    \end{minipage}
    \hfill
    \begin{minipage}[!h]{0.50\hsize}\centering
        \begin{center}\resizebox{0.8\textwidth}{!}{%
                %\caption{Лог работы программы.}
                \begin{tabular}{|c@{\hspace{5mm}}|c@{\hspace{5mm}}|c@{\hspace{5mm}}|c@{\hspace{5mm}}|c@{\hspace{5mm}}|c|}
                    \hline
                    $\alpha$ & $\beta$ & $\rho$ & Длина & Разница \\
                    \hline
                    0.3  & 0.7  & 0.7  & 6986  & 0     \\
                    0.3  & 0.7  & 0.8  & 6992  & 6     \\
                    0.3  & 0.7  & 0.9  & 6992  & 6     \\
                    0.3  & 0.7  & 1    & 6986  & 0     \\
                    0.4  & 0.6  & 0    & 6986  & 0     \\
                    0.4  & 0.6  & 0.1  & 6992  & 6     \\
                    0.4  & 0.6  & 0.2  & 6986  & 0     \\
                    0.4  & 0.6  & 0.3  & 6986  & 0     \\
                    0.4  & 0.6  & 0.4  & 6986  & 0     \\
                    0.4  & 0.6  & 0.5  & 6992  & 6     \\
                    0.4  & 0.6  & 0.6  & 6992  & 6     \\
                    0.4  & 0.6  & 0.7  & 6986  & 0     \\
                    0.4  & 0.6  & 0.8  & 7139  & 153   \\
                    0.4  & 0.6  & 0.9  & 6986  & 0     \\
                    0.4  & 0.6  & 1    & 6992  & 6     \\
                    0.5  & 0.5  & 0    & 7139  & 153   \\
                    0.5  & 0.5  & 0.1  & 6986  & 0     \\
                    0.5  & 0.5  & 0.2  & 6986  & 0     \\
                    0.5  & 0.5  & 0.3  & 7139  & 153   \\
                    0.5  & 0.5  & 0.4  & 6986  & 0     \\
                    0.5  & 0.5  & 0.5  & 6986  & 0     \\
                    0.5  & 0.5  & 0.6  & 6986  & 0     \\
                    0.5  & 0.5  & 0.7  & 6986  & 0     \\
                    0.5  & 0.5  & 0.8  & 6986  & 0     \\
                    0.5  & 0.5  & 0.9  & 6986  & 0     \\
                    0.5  & 0.5  & 1    & 6986  & 0     \\
                    0.6  & 0.4  & 0    & 7139  & 153   \\
                    0.6  & 0.4  & 0.1  & 6992  & 6     \\
                    0.6  & 0.4  & 0.2  & 6986  & 0     \\
                    0.6  & 0.4  & 0.3  & 6986  & 0     \\
                    0.6  & 0.4  & 0.4  & 7139  & 153   \\
                    0.6  & 0.4  & 0.5  & 6992  & 6     \\
                    0.6  & 0.4  & 0.6  & 6986  & 0     \\
                    0.6  & 0.4  & 0.7  & 6986  & 0     \\
                    0.6  & 0.4  & 0.8  & 6986  & 0     \\
                    0.6  & 0.4  & 0.9  & 6992  & 6     \\
                    0.6  & 0.4  & 1    & 6986  & 0     \\
                    0.7  & 0.3  & 0    & 6986  & 0     \\
                    0.7  & 0.3  & 0.1  & 6986  & 0     \\
                    0.7  & 0.3  & 0.2  & 6986  & 0     \\
                    0.7  & 0.3  & 0.3  & 7139  & 153   \\
                    \hline
                \end{tabular}}
                %\label{T:log}
        \end{center}
    \end{minipage}
\end{table}
\clearpage
\begin{table}[!h]
    \begin{center}
        \begin{tabular}{|c@{\hspace{7mm}}|c@{\hspace{7mm}}|c@{\hspace{7mm}}|c@{\hspace{7mm}}|c@{\hspace{7mm}}|c|}
            \hline
            $\alpha$ & $\beta$ & $\rho$ & Длина & Разница \\
            \hline
            0.7  & 0.3  & 0.4  & 7165  & 179   \\
            0.7  & 0.3  & 0.5  & 7139  & 153   \\
            0.7  & 0.3  & 0.6  & 6992  & 6     \\
            0.7  & 0.3  & 0.7  & 6992  & 6     \\
            0.7  & 0.3  & 0.8  & 6986  & 0     \\
            0.7  & 0.3  & 0.9  & 6992  & 6     \\
            0.7  & 0.3  & 1    & 6986  & 0     \\
            0.8  & 0.2  & 0    & 7139  & 153   \\
            0.8  & 0.2  & 0.1  & 7562  & 576   \\
            0.8  & 0.2  & 0.2  & 6992  & 6     \\
            0.9  & 0.1  & 0.2  & 6992  & 6     \\
            0.9  & 0.1  & 0.3  & 6986  & 0     \\
            0.9  & 0.1  & 0.4  & 7139  & 153   \\
            0.9  & 0.1  & 0.5  & 7329  & 343   \\
            0.9  & 0.1  & 0.6  & 7217  & 231   \\
            0.9  & 0.1  & 0.7  & 7139  & 153   \\
            0.9  & 0.1  & 0.8  & 7217  & 231   \\
            0.9  & 0.1  & 0.9  & 7376  & 390   \\
            0.9  & 0.1  & 1    & 6986  & 0     \\
            1    & 0    & 0    & 8531  & 1545  \\
            1    & 0    & 0.1  & 8588  & 1602  \\
            1    & 0    & 0.2  & 6986  & 0     \\
            1    & 0    & 0.3  & 7720  & 734   \\
            1    & 0    & 0.4  & 7554  & 568   \\
            1    & 0    & 0.5  & 6992  & 6     \\
            1    & 0    & 0.6  & 7920  & 934   \\
            1    & 0    & 0.7  & 7217  & 231   \\
            1    & 0    & 0.8  & 7874  & 888   \\
            1    & 0    & 0.9  & 7446  & 460   \\
            1    & 0    & 1    & 8119  & 1133  \\
            \hline
        \end{tabular}
    \end{center}
\end{table}
\clearpage


\section{Вывод}

В данном разделе было произведено сравнение количества затраченного времени вышеизложенных алгоритмов.

В результате сравнения алгоритма полного перебора и муравьиного алгоритма по времени из таблицы \ref{tbl:time} были получены следующие результаты:
\begin{itemize}
    \item при относительно небольших размерах матрицы смежности (а именно от 2 до 8) алгоритм полного перебора работает быстрее (при размере 2 --- $\approx$ в 1000 раз, при размере 8 --- $\approx$ в 10 раз);
    \item при размере матрицы смежности равном 9, время работы алгоритма полного перебора становится сопоставимым с временем работы муравьиного алгоритма.
    \item при размерах матрицы смежности больших 9, время работы алгоритма полного перебора начинает резко возрастать, и становится при размере 10 в 10 раз медленнее муравьиного алгоритма.
\end{itemize}

Исходя из проведенных исследований, можно сделать вывод, что муравьиный алгоритм решения задачи коммивояжера выигрывает у алгоритма полного перебора при размерностях анализируемых графов равных 9 и больше вершин. В случае, если количество вершин в графе меньше 9, лучше воспользоваться алгоритмом полного перебора.

На основе проведенной параметризации для двух классов данных можно сделать следующие выводы:
\begin{itemize}
    \item Для класса данных №1, содержащего приблизительно равные значения, наилучшими наборами стали $(\alpha = 0.5, \beta = 0.5, \rho = \text{любое})$, так как они показали наиболее стабильные результаты, равные эталонному значению оптимального пути, равного 52 единицам.
    \item Для класса данных №2, содержащего различные значения, наилучшими наборами стали $(\alpha = 0.5, \beta = 0.5, \rho = \text{любое})$. При этих параметрах, количество найденных эталонных оптимальных путей составило 8 единиц.
\end{itemize}