\chapter*{Введение}
\addcontentsline{toc}{chapter}{Введение}

В данной лабораторной работе будут рассмотрены алгоритмы умножения матриц.

Матрицы A и B могут быть перемножены, если число столбцов матрицы A равно числу строк B. 

Алгоритм Винограда -- алгоритм умножения квадратных матриц, предложенный в 1987 году Д. Копперсмитом и Ш. Виноградом. В исходной версии асимптотическая сложность алгоритма составляла $O(n^{2,3755})$, где $n$ -- размер стороны матрицы. Алгоритм Винограда, с учетом серии улучшений и доработок в последующие годы, обладает лучшей асимптотикой среди известных алгоритмов умножения матриц.

На практике алгоритм Винограда не используется \cite{alg}, так как он имеет очень большую константу пропорциональности и начинает выигрывать в быстродействии у других известных алгоритмов только для матриц, размер которых превышает память современных компьютеров.

В настоящее время умножение матриц активно используется в компьютерной графике, криптографии.

Целью данной работы является изучение, программная реализация, а также экспериментальное сравнение следующих алгоритмов:
\begin{itemize}
	\item стандартный алгоритм умножения матриц;
	\item алгоритм Винограда;
	\item оптимизированный алгоритм Винограда.
\end{itemize}

Для достижения данной цели необходимо решить следующие задачи:
\begin{itemize}
	\item изучить и реализовать стандартный алгоритм умножения матриц;
	\item изучить и реализовать алгоритм Винограда умножения матриц;
	\item оптимизировать алгоритм Винограда умножения матриц;
	\item оценить трудоемкость реализаций алгоритмов умножения матриц теоретически;
	\item сравнить временные характеристики вышеизложенных алгоритмов экспериментально.
\end{itemize}