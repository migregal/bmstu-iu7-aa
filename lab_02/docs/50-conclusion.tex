\chapter*{Заключение}
\addcontentsline{toc}{chapter}{Заключение}

Среди рассмотренных алгоритмов наиболее эффективным оказался алгоритм Винограда, однако нехначительное (порядка $ 10\% $) улучшение характеристик по времени повлекло за собой дополнительные затраты по памяти ($n + m$, где $n$ и $m$ - соответствующие размерности матриц).

Алгоритм Винограда становится тем эффективнее по времени, чем большие размерности матриц подаются на вход алгоритма.

В связи с вышеууказанным, оптимизированный алгоритм Винограда является предпочтительным при обработке больших матриц, однако, при строгих ограничениях на затраты по памяти, простой (или наивный) алгоритм умножения матриц является более предпочтительным, т.к. расходует лишь $n \cdot m$ памяти для хранения результирующей матрицы.


В рамках выполнения работы решены следующие задачи.
\begin{enumerate}
	\item изучен и реализован стандартный алгоритм умножения матриц;
	\item изучен и реализован алгоритм Винограда умножения матриц;
	\item оптимизирован алгоритм Винограда умножения матриц;
	\item произведена оценка трудоемкости реализаций алгоритмов умножения матриц;
	\item произведены сравнения временных характеристик вышеизложенных алгоритмов.
\end{enumerate}
