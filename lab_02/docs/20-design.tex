\chapter{Конструкторская часть}

В данном разделе будут рассмотрены схемы алгоритмов умножения матриц и модель вычислений.

\section{Схемы}
На рисунке \ref{img:std_mul} представлена схема стандартного алгоритма умножения матриц.

\imgw{std_mul}{ht!}{0.8\textwidth}{Схема стандартного алгоритма умножения матриц}

\clearpage

На рисунках \ref{img:win_mul} -- \ref{img:win_mul_2} представлена схема алгоритма Винограда умножения матриц.

\imgw{win_mul}{ht!}{0.8\textwidth}{Схема алгоритма Винограда умножения матриц. Часть 1}

\imgw{win_mul_2}{ht!}{0.8\textwidth}{Схема алгоритма Винограда умножения матриц. Часть 2}

\clearpage
\section{Модель вычислений}

Для последующего вычисления трудоемкости введём модель вычислений:

\begin{itemize}
	\item операции из списка (\ref{for:opers}) имеют трудоемкость 1;
	\begin{equation}
	\label{for:opers}
	+, -, =, +=, -=, ==, !=, <, >, <=, >=, >>, <<, [], ++, {-}-
	\end{equation}
	\item операции из списка (\ref{for:opers}) имеют трудоемкость 2;
	\begin{equation}
	\label{for:opers2}
	*,  /, \%, /=, *=
	\end{equation}
	\item трудоемкость оператора выбора if условие then A else B рассчитывается, как (\ref{for:if});
	\begin{equation}
	\label{for:if}
	f_{if} = f_{\text{условия}} +
	\begin{cases}
	f_A, & \text{если условие выполняется,}\\
	f_B, & \text{иначе.}
	\end{cases}
	\end{equation}
	\item трудоемкость цикла рассчитывается, как (\ref{for:for}).
	\begin{equation}
	\label{for:for}
	f_{for} = f_{\text{инициализации}} + f_{\text{сравнения}} + N(f_{\text{тела}} + f_{\text{инкремента}} + f_{\text{сравнения}})
	\end{equation}
	%\item трудоемкость вызова функции равна 0.
\end{itemize}

\section{Трудоёмкость алгоритмов}

\subsection{Стандартный алгоритм умножения матриц}

Трудоёмкость стандартного алгоритма умножения матриц состоит из:
\begin{itemize}
	\item внешнего цикла по $i \in [1..n]$, трудоёмкость которого: 
	
	$f = 2 + n \cdot (2 + f_{body})$;
	\item цикла по $j \in [1..m]$, трудоёмкость которого: $f = 2 + m \cdot (2 + f_{body})$;
	\item скалярного умножения двух векторов - цикл по $k \in [1..q]$, трудоёмкость которого: $f = 2 + 14q$;
\end{itemize}

Трудоёмкость стандартного алгоритма равна трудоёмкости внешнего цикла, можно вычислить ее, подставив циклы тела (\ref{for:base}):
\begin{equation}
	\label{for:base}
	f_{base} = 2 + n \cdot (4 + m \cdot (4 + 14q)) = 2 + 4 \cdot n + 4 \cdot nm + 14 \cdot nmq \approx 14 \cdot nmq
\end{equation}

\subsection{Алгоритм Винограда}

Трудоёмкость алгоритма Винограда состоит из:
\begin{enumerate}
	\item  формирования массива сумм произведений пар соседних элементов строк матрицы A:
	\begin{equation}
	\begin{array}{c}
	f_1 = 2 + n  \cdot(2 + 4 + q/2  \cdot (4 + 1 + 6 + 3  \cdot 2 + 2)) =  \\
		19/2 \cdot nm + 6 \cdot n + 2 ;
	\end{array}
	\end{equation}
	
	\item  формирования массива сумм произведений пар соседних элементов строк матрицы B:
	\begin{equation}
	f_2 = 19/2 \cdot mq + 6 \cdot m + 2;
	\end{equation}
	
	\item цикла заполнения ячеек матрицы C для чётных размеров:
	\begin{equation}
	\begin{array}{c}
	f_3 = 2 + n \cdot (2 + 2 + m \cdot (2 + 7 + 4 + q/2 \cdot (4 + 1 + 12 + 5 + 5 \cdot 2))) = \\
		32/2 \cdot nqm + 13 \cdot nm + 4 \cdot n + 2 ;
	\end{array}
	\end{equation}
	
	\item цикла для дополнения умножения суммой последних нечётных строки и столбца, если общий размер нечётный (\ref{for:last}):
	\begin{equation}
	\label{for:last}
	f_4 =  3 + \begin{cases}
	0, & \text{чётная,}\\
	2 +4 \cdot n + 13 \cdot nm), & \text{иначе.}
	\end{cases}
	\end{equation}
\end{enumerate}

Итого, для худшего (нечётный размер матриц) и лучшего (чётный размер матриц) случаев:
\begin{equation}
\label{for:bad}
f_{winograd} = 16 \cdot nqm.
\end{equation}

\subsection{Оптимизированный алгоритм Винограда}

Из рисунков \ref{img:win_mul} -- \ref{img:win_mul_2} можно заметить, что для алгоритма Винограда худшим случаем являются матрицы с нечётным общим размером, а лучшим - с чётным, так как отпадает необходимость в последнем цикле.
Данный алгоритм можно оптимизировать:
\begin{enumerate}
	\item заменив выражения вида \texttt{a = a + ...} на \texttt{a += ...};
	\item введя промежуточный буфер для подсчета промежуточных значений;
	\item сделав в циклах по k шаг 2, избавившись тем самым от двух операций умножения на каждую итерацию.
\end{enumerate}

Трудоёмкость оптимизированного алгоритма Винограда состоит из:
\begin{enumerate}
	\item  формирования массива сумм произведений пар соседних элементов строк матрицы A:
	\begin{equation}
	\begin{array}{c}
	f_1 = 11/2 \cdot nq + 4 \cdot n + 2 ;
	\end{array}
	\end{equation}
	
	\item  формирования массива сумм произведений пар соседних элементов строк матрицы B:
	\begin{equation}
	f_2 = 11/2 \cdot mq + 4 \cdot m + 2;
	\end{equation}
	
	\item цикла заполнения ячеек матрицы C для чётных размеров:
	\begin{equation}
	\begin{array}{c}
	f_3 = 2 + n \cdot (2 + 2 + m \cdot (2 + 6 + 2 + q/2 \cdot (2 + 1 + 10 + 4 + 1 \cdot 2))) = \\
		17/2 \cdot nqm + 13 \cdot nm + 4 \cdot n + 2 ;
	\end{array}
	\end{equation}
	
	\item цикла для дополнения умножения суммой последних нечётных строки и столбца, если общий размер нечётный (\ref{for:last_2}):
	\begin{equation}
	\label{for:last_2}
	f_4 =  2 + \begin{cases}
	0, & \text{чётная,}\\
	2 +4 \cdot n + 13 \cdot nm), & \text{иначе.}
	\end{cases}
	\end{equation}
\end{enumerate}

Итого, для худшего (нечётный размер матриц) и лучшего (чётный размер матриц) случаев:
\begin{equation}
\label{for:bad_2}
f_{winograd} = 17/2 \cdot nqm  \approx 8.5  \cdot nqm .
\end{equation}

\section{Вывод}

В данном разделе были рассмотрены схемы (рис. \ref{img:std_mul} -- \ref{img:win_mul_2}) алгоритмов умножения матриц.
Оценены их трудоемкости в лучшем и худшем случаях.