\chapter{Технологическая часть}

\section{Выбор языка программирования}

В качестве языка программирования для реализации данной лабораторной работы был выбран язык Golang \cite{golang}. Данный выбор обусловлен тем, что я имею некоторый опыт разработки на нем, а так же наличием у языка встроенных высокоточных средств тестирования и анализа разработанного ПО.

\section{Требования к программному обеспечению}

Входными данными являются две матрицы $A$ и $B$.
Количество столбцов матрицы A должно быть равно количеству строк матрицы $B$. 

На выходе получается результат умножения введенных пользователем матриц.

\section{Сведения о модулях программы}

Данная программа разбита на модули:
\begin{itemize}
    \item \texttt{main.go} - файл, содержащий точку входа в программу. В нем происходит общение с пользователем и вызов алгоритмов;
    \item \texttt{simple\_mult.go} -- Файл содержит реализацию простого алгоритма умножения матриц.
    \item \texttt{winograd.go} -- Файл содержит реализацию алгоритма умножения матриц по Винограду.
    \item \texttt{winograd\_imp.go} -- Файл содержит улучшенную реализацию алгоритма умножения матриц по Винограду.
	\item \texttt{utils.go} -- Файл содержит различные функции для вычислений. матриц.
\end{itemize}

На листингах \ref{lst:main} -- \ref{lst:winograd_imp_2} представлен код программы.

\clearpage

\listingfile{main.go}{main}{Go}{Основной файл программы main}{linerange={11-44}}

\clearpage

\listingfile{utils.go}{utils}{Go}{Различные функции для вычислений}{linerange={8-42}}

\clearpage

\listingfile{simple_mult.go}{simple_mult}{Go}{Простое умножение}{linerange={3-15}}

\listingfile{winograd.go}{winograd_1}{Go}{Алгоритм Винограда. Часть 1}{linerange={3-26}}

\clearpage

\listingfile{winograd.go}{winograd_2}{Go}{Алгоритм Винограда. Часть 2}{linerange={28-48}}

\listingfile{winograd_imp.go}{winograd_imp_1}{Go}{Оптимизированный алгоритм Винограда. Часть 1}{linerange={3-26}}

\clearpage

\listingfile{winograd_imp.go}{winograd_imp_2}{Go}{Оптимизированный алгоритм Винограда. Часть 2}{linerange={27-48}}

\section{Вывод}

В данном разделе были реализованны вышеописанные алголритмы. 

Было разработано программное обеспечение, удовлетворяющее предъявляемым требованиям. Так же были представлены соответствующие листинги \ref{lst:main} -- \ref{lst:winograd_imp_2} с кодом программы.