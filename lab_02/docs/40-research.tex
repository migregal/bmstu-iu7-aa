\chapter{Экспериментальная часть}

В данном разделе будет проведено функциональное тестирование разработанного программного обеспечения. Так же будет произведено измерение временных характеристик каждого из реализованных алгоритмов. 

Для проведения подобных экспериментов на языке программирования \texttt{Golang} \cite{golang}, используется специальный пакет \texttt{testing} \cite{gotest}. Данный пакет предоставляет инструменты для измерения процессорного времени и объема памяти, использованных конкретным алгоритмом в ходе проведения 
эксперимента.

\section{Технические характеристики}

Технические характеристики устройства, на котором выполнялось исследование:

\begin{itemize}
	\item Процессор: Intel Core™ i5-8250U \cite{i5} CPU @ 1.60GHz.
	\item Память: 32 GiB.
	\item Операционная система: Ubuntu \cite{ubuntu} Linux \cite{linux} 20.04 64-bit.

\end{itemize}

Исследование проводилось на ноутбуке, включенном в сеть электропитания. Во время тестирования ноутбук был нагружен только встроенными приложениями окружения рабочего стола, окружением рабочего стола, а также непосредственно системой тестирования.

\section{Тестирование}

В данном разделе будет приведена таблица с тестами (таблица \ref{table:ref1}).

\begin{table}[h!]
	\begin{center}
	\caption{Таблица тестов}
	\label{table:ref1}
		\begin{tabular}{c@{\hspace{7mm}}c@{\hspace{7mm}}c@{\hspace{7mm}}c@{\hspace{7mm}}c@{\hspace{7mm}}c@{\hspace{7mm}}}
			\hline
			Первая матрица & Вторая матрица & Ожидаемый результат \\ \hline
			\vspace{4mm}
			$\begin{pmatrix}
			2
			\end{pmatrix}$ &
			$\begin{pmatrix}
			2
			\end{pmatrix}$ &
			$\begin{pmatrix}
			4
			\end{pmatrix}$ \\
			\vspace{2mm}
			\vspace{2mm}			
			$\begin{pmatrix}
			2 \\
			1 
			\end{pmatrix}$ &
			$\begin{pmatrix}
			1 & 2
			\end{pmatrix}$ &
			$\begin{pmatrix}
			2 & 4\\
			1 & 2
			\end{pmatrix}$\\
			\vspace{2mm}
			\vspace{2mm}
			$\begin{pmatrix}
			1 & 2 & 2\\
			1 & 2 & 2
			\end{pmatrix}$ &
			$\begin{pmatrix}
			1 & 2\\
			1 & 2\\
			1 & 2
			\end{pmatrix}$ &
			$\begin{pmatrix}
			5 & 10\\
			5 & 10
			\end{pmatrix}$ \\
			\vspace{2mm}
			\vspace{2mm}
			$\begin{pmatrix}
			1 & -2 & 3\\
			1 & 2 & 3\\
			1 & 2 & 3
			\end{pmatrix}$ &
			$\begin{pmatrix}
			-1 & 2 & 3\\
			1 & 2 & 3\\
			1 & 2 & 3
			\end{pmatrix}$ &
			$\begin{pmatrix}
			0 & 4 & 6\\
			4 & 12 & 18\\
			4 & 12 & 18
			\end{pmatrix}$\\
		\end{tabular}
	\end{center}
\end{table}
\newpage
При проведении функционального тестирования, полученные результаты работы программы совпали с ожидаемыми. Таким образом, функциональное тестирование пройдено успешно.

\section{Временные характеристики}

Для сравнения возьмем квадратные матрицы размерностью [100, 200, 300, \dots, 800]. 

Результаты замеров по результатам экспериментов приведены в Таблице \ref{tbl:time_even}.

\begin{table}[ht]
	\small
	\begin{center}
		\caption{Замер времени для матриц, размером от 100 до 800 элементов}
		\label{tbl:time_even}
		\begin{tabular}{|c|c|c|c|}
			\hline
			& \multicolumn{3}{c|}{\bfseries Время, нс} \\ \cline{1-4}
			\bfseries Размерность матрицы, эл. & \bfseries Простой & \bfseries Виноград & \bfseries Оптимизированный Виноград
			\csvreader{inc/csv/time_even.csv}{}
			{\\\hline \csvcoli&\csvcolii&\csvcoliii&\csvcoliv}
			\\\hline
		\end{tabular}
	\end{center}
\end{table}

\begin{figure}[ht]
	\centering
	\begin{tikzpicture}
		\begin{axis}[
			axis lines=left,
			xlabel={Размер массива, эл},
			ylabel={Время, мс},
			legend pos=north west,
			ymajorgrids=true
		]
			\addplot table[x=len,y=simple,col sep=comma] {inc/csv/time_even.csv};
			\addplot table[x=len,y=winograd,col sep=comma] {inc/csv/time_even.csv};
			\addplot table[x=len,y=winograd_imp,col sep=comma] {inc/csv/time_even.csv};
			\legend{Простой, Виноград, Виноград с оптимизациями}
		\end{axis}
	\end{tikzpicture}
	\captionsetup{justification=centering}
	\caption{Временные характеристики на четных размерах матриц}
	\label{plt:time_even_cmp}
\end{figure}

Отдельно сравним временные характеристики при нечетных размерностях матриц [101, 201, 301, \dots, 801].

Результаты замеров по результатам экспериментов приведены в Таблице \ref{tbl:time_odd}.

\begin{table}[ht]
	\small
	\begin{center}
		\caption{Замер времени для матриц, размером от 101 до 801 элементов}
		\label{tbl:time_odd}
		\begin{tabular}{|c|c|c|c|}
			\hline
			& \multicolumn{3}{c|}{\bfseries Время, нс} \\ \cline{1-4}
			\bfseries Размерность матрицы, эл. & \bfseries Простой & \bfseries Виноград & \bfseries Оптимизированный Виноград
			\csvreader{inc/csv/time_odd.csv}{}
			{\\\hline \csvcoli&\csvcolii&\csvcoliii&\csvcoliv}
			\\\hline
		\end{tabular}
	\end{center}
\end{table}

\begin{figure}[ht]
	\centering
	\begin{tikzpicture}
		\begin{axis}[
			axis lines=left,
			xlabel={Размер массива, эл},
			ylabel={Время, мс},
			legend pos=north west,
			ymajorgrids=true
		]
			\addplot table[x=len,y=simple,col sep=comma] {inc/csv/time_odd.csv};
			\addplot table[x=len,y=winograd,col sep=comma] {inc/csv/time_odd.csv};
			\addplot table[x=len,y=winograd_imp,col sep=comma] {inc/csv/time_odd.csv};
			\legend{Простой, Виноград, Виноград с оптимизациями}
		\end{axis}
	\end{tikzpicture}
	\captionsetup{justification=centering}
	\caption{Временные характеристики на нечетных размерах матриц}
	\label{plt:time_odd_cmp}
\end{figure}

\newpage
\section{Вывод}
В данном разделе было произведено сравнение количества затраченного вре­мени вышеизложенных алгоритмов.
Наименее затратным по времени оказался оптимизированный алгоритм Винограда. Но при этом ему дополнительно требуется $n+m$ памяти под результат (т.е. $n+m$ прибавляется к тому количеству памяти, которое выделяется под результат в реализации стандарного алгоритма умножения матриц - $n \cdot m$).

Время работы реализации алгоритма Винограда незначительно меньше времени работы реализации простого алгоритма умножения, однако при размере $800 × 800$ время вычислений реализации алгоритма Винограда на 0.3 секунды меньше, нежели у реализации простого алгоритма (что составляет в данном случае порядка $10\%$).

Такой результат совпадает с теоретически полученными оценками трудоемкости алгоритмов 
