\chapter{Аналитическая часть}

\section{Описание алгоритмов}

\textbf{Сортировка пузырьком}

Алгоритм пузырьковой сортировки совершает несколько проходов по списку. 
При каждом проходе происходит сравнение соседних элементов. Если порядок соседних элементов неправильный, то они меняются местами. Каждый проход начинается с начала списка.

\textbf{Сортировка вставками}

Сортировка вставками — алгоритм сортировки, в котором элементы входной последовательности просматриваются по одному, и каждый новый поступивший элемент размещается в подходящее место среди ранее упорядоченных элементов \cite{knut}.

В начальный момент отсортированная последовательность пуста. На каждом шаге алгоритма выбирается один из элементов входных данных и помещается на нужную позицию в уже отсортированной последовательности до тех пор, пока набор входных данных не будет исчерпан. В любой момент времени в отсортированной последовательности элементы удовлетворяют требованиям к выходным данным алгоритма \cite{kormen}.

\textbf{Сортировка выбором}

Алгоритм сортировки выбором совершает несколько проходов по списку. При каждом проходе выбирается минимальный из еще не отсортированных элементов и обменивается с первым элементом не отсортированной области. В следующем проходе рассмотренный элемент не участвует, сортируется только оставшийся хвост.

Для реализации устойчивости алгоритма необходимо минимальный элемент непосредственно вставлять в первую не отсортированную позицию, не меняя порядок остальных элементов.

\section{Вывод}

Пузырьковая сортировка сравнивает элементы попарно, переставляя между собой элементы тех пар, порядок в которых нарушен.
Сортировка вставками, сортирует список, вставляя очередной элемент в нужное место уже отсортированного списка.
Быстрая сортировка определяет опорный элемент и далее переставляет элементы, относительно выбранного элемента.

Были рассмотрены основополагающие материалы, которые в дальнейшем потребуются при реализации алгоритмов сортировки. 