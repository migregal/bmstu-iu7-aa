\chapter{Конструкторская часть}

В данном разделе будут рассмотрены схемы вышеизложенных алгоритмов.

\section{Схемы}

На рисунке \ref{img:bubble_sort} представлена схема алгоритма сортировки пузырьком.

\imgw{bubble_sort}{ht!}{0.7\textwidth}{Схема алгоритма сортировки пузырьком}

\clearpage

На рисунке \ref{img:insertion_sort} представлена схема алгоритма сортировки вставками.

\imgw{insertion_sort}{ht!}{0.7\textwidth}{Схема алгоритма сортировки вставками}

\clearpage

На рисунке \ref{img:selection_sort} представлена схема алгоритма сортировки выбором.

\imgw{selection_sort}{ht!}{0.7\textwidth}{Схема алгоритма сортировки выбором}

\clearpage

\section{Модель вычислений}

Для последующего вычисления трудоемкости введём модель вычислений:

\begin{itemize}
	\item операции из списка (\ref{for:opers}) имеют трудоемкость 1;
	\begin{equation}
	\label{for:opers}
	+, -, /, \%, ==, !=, <, >, <=, >=, [], ++, {-}-
	\end{equation}
	\item трудоемкость оператора выбора if условие then A else B рассчитывается, как (\ref{for:if});
	\begin{equation}
	\label{for:if}
	f_{if} = f_{\text{условия}} +
	\begin{cases}
	f_A, & \text{если условие выполняется,}\\
	f_B, & \text{иначе.}
	\end{cases}
	\end{equation}
	\item трудоемкость цикла рассчитывается, как (\ref{for:for});
	\begin{equation}
	\label{for:for}
	f_{for} = f_{\text{инициализации}} + f_{\text{сравнения}} + N(f_{\text{тела}} + f_{\text{инкремента}} + f_{\text{сравнения}})
	\end{equation}
	\item трудоемкость вызова функции равна 0.
\end{itemize}

\section{Трудоёмкость алгоритмов}

Пусть размер массивов во всех вычислениях обозначается как $N$.

\subsection{Алгоритм сортировки пузырьком}

Трудоёмкость алгоритма сортировки пузырьком состоит из:
\begin{itemize}
	\item трудоёмкость сравнения и инкремента внешнего цикла $i \in [1..N)$ (\ref{for:bubble_outer}):
	\begin{equation}
	\label{for:bubble_outer}
	f_{i} = 2 + 2(N - 1)
	\end{equation}
	\item суммарная трудоёмкость внутренних циклов, количество итераций которых меняется в промежутке $[1..N-1]$ (\ref{for:bubble_inner}):
	\begin{equation}
	\label{for:bubble_inner}
	f_{j} = 3(N - 1) + \frac{N \cdot (N - 1)}{2} \cdot (3 + f_{if})
	\end{equation}
	\item трудоёмкость условия во внутреннем цикле (\ref{for:bubble_if}):
	\begin{equation}
	\label{for:bubble_if}
	f_{if} = 4 + \begin{cases}
	0, & \text{в лучшем случае}\\
	9, & \text{в худшем случае}\\
	\end{cases}
	\end{equation}
\end{itemize}

Трудоёмкость в \textbf{лучшем} случае (\ref{for:bubble_best}):
\begin{equation}
\label{for:bubble_best}
f_{best} = \frac{7}{2} N^2 + \frac{3}{2} N - 3 \approx \frac{7}{2} N^2 = O(N^2)
\end{equation}

Трудоёмкость в \textbf{худшем} случае (\ref{for:bubble_worst}):
\begin{equation}
\label{for:bubble_worst}
f_{worst} =  8N^2 - 8N - 3 \approx 8N^2 = O(N^2)
\end{equation}

\subsection{Алгоритм сортировки вставками}

Трудоёмкость алгоритма сортировки вставками состоит из:
\begin{itemize}
	\item трудоёмкость сравнения и инкремента внешнего цикла $i \in [1..N)$ (\ref{for:isort_outer}):
	\begin{equation}
	\label{for:isort_outer}
	f_{i} = 2 + 2(N - 1)
	\end{equation}
	\item суммарная трудоёмкость внутренних циклов, количество итераций которых меняется в промежутке $[1..N-1]$ (\ref{for:isort_inner}):

	\begin{equation}
	\label{for:isort_inner}
	f_{if} = 4 + \begin{cases}
		0, & \text{в лучшем случае}\\
		3(N - 1) + \frac{N \cdot (N - 1)}{2} \cdot (3 + f_{if}), & \text{в худшем случае}\\
	\end{cases}
	\end{equation}

	\item трудоёмкость условия во внутреннем цикле (\ref{for:isort_if}):
	\begin{equation}
	\label{for:isort_if}
	f_{if} = 4 + \begin{cases}
	0, & \text{в лучшем случае}\\
	9, & \text{в худшем случае}\\
	\end{cases}
	\end{equation}
\end{itemize}

Трудоёмкость в \textbf{лучшем} случае (\ref{for:isort_best}):
\begin{equation}
\label{for:isort_best}
f_{best} = 13N - 10 \approx 13N = O(N)
\end{equation}

Трудоёмкость в \textbf{худшем} случае (\ref{for:isort_worst}):
\begin{equation}
\label{for:isort_worst}
f_{worst} = 4.5N^2 + 10N - 13 \approx 4N^2 = O(N^{2})
\end{equation}

\subsection{Алгоритм сортировки выбором}

Трудоемкость сортировки выбором в лучшем случае: $O(N^2)$ \cite{knut}.

Трудоемкость сортировки выбором в худшем случае: $O(N^2)$ \cite{knut}.

\section{Вывод}

На основе теоретических данных, полученных из аналитического раздела,
были построены схемы (рисунки \ref{img:bubble_sort} -- \ref{img:selection_sort}) трех алгоритмов сортировки. Оценены их трудоемкости в лучшем и худшем случаях. Все алгоритмы в худшем случае обладают квадратичной сложностью. А в лучшем случае меньше всего сложность у алгоритма сортировки вставками.