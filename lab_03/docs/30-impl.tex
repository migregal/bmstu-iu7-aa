\chapter{Технологическая часть}

В данном разделе приведены требования к программному обеспечению, средства реализации и листинги кода.

\section{Выбор средств реализации}

В качестве языка программирования для реализации данной лабораторной работы был выбран язык Golang \cite{golang}. Данный выбор обусловлен тем, что я имею некоторый опыт разработки на нем, а так же наличием у языка встроенных высокоточных средств тестирования и анализа разработанного ПО. 

\section{Требования к программному обеспечению}

Входными данными являются:
\begin{itemize}
    \item размерность массива $n$;
    \item $n$ элементов массива.
\end{itemize}

На выходе получается отсортированный массив.


\section{Сведения о модулях программы}

Данная программа разбита на следующие модули:
\begin{itemize}
\item \texttt{main.go} -- Файл, содержащий точку входа в программу. В нем происходит общение с пользователем и вызов алгоритмов.
\item \texttt{bubble.go} -- Файл содержит реализацию сортировки пузырьком.
\item \texttt{insertion.go} -- Файл содержит реализацию сортировки вставками.
\item \texttt{selection.go} -- Файл содержит реализацию сортировки выбором.
\item \texttt{utils.go} -- Файл содержит различные функции для вычислений.
\end{itemize}

В листингах \ref{lst:bubble}--\ref{lst:selection} представлены исходные коды разобранных ранее алгоритмов.
\clearpage

\listingfile{main.go}{main}{Go}{Основной файл программы main}{linerange={9-34}}

\listingfile{utils.go}{utils}{Go}{Различные функции для вычислений}{linerange={5-17}}

\clearpage

\listingfile{bubble.go}{bubble}{Go}{Сортировка пузырьком}{linerange={3-13}}

\listingfile{insertion.go}{insertion}{Go}{Сортировка вставками}{linerange={3-13}}

\listingfile{selection.go}{selection}{Go}{Сортировка выбором}{linerange={3-14}}

\section{Вывод}

Были реализованы спроектированные алгоритмы: сортировки пузырьком, вставками и выбором.
