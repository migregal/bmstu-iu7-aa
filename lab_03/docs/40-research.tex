\chapter{Экспериментальная часть}

В данном разделе будет проведено функциональное тестирование разработанного программного обеспечения. Так же будет произведено измерение временных характеристик и характеристик по памяти каждого из реализованных алгоритмов. 

Для проведения подобных экспериментов на языке программирования \texttt{Golang} \cite{golang}, используется специальный пакет \texttt{testing} \cite{gotest}. Данный пакет предоставляет инструменты для измерения процессорного времени и объема памяти, использованных конкретным алгоритмом в ходе проведения 
эксперимента.

\section{Тестирование}

В таблице \ref{tabular:functional_test} приведены функциональные тесты для алгоритмов сортировки.

\begin{table}[h]
    \small
	\begin{center}
		\caption{\label{tabular:functional_test} Функциональные тесты}
		\begin{tabular}{|l|l|l|l|l|}
			\hline
			\bfseries Тип случая & \bfseries Исходный массив & \bfseries Пузырек & \bfseries Вставками & \bfseries Выбором
			\csvreader{inc/csv/functional_test.csv}{}
			{\\\hline \csvcoli&\csvcolii&\csvcoliii&\csvcoliv&\csvcolv}
			\\\hline
		\end{tabular}
	\end{center}
\end{table}

При проведении функционального тестирования, полученные результаты работы программы совпали с ожидаемыми. Таким образом, функциональное тестирование пройдено успешно.

\section{Технические характеристики}

Технические характеристики устройства, на котором выполнялось исследование:

\begin{itemize}
	\item процессор: Intel Core™ i5-8250U \cite{i5} CPU @ 1.60GHz;
	\item память: 32 GiB;
	\item операционная система: Ubuntu \cite{ubuntu} Linux \cite{linux} 20.04 64-bit;

\end{itemize}

Исследование проводилось на ноутбуке, включенном в сеть электропитания. Во время тестирования ноутбук был нагружен только встроенными приложениями окружения рабочего стола, окружением рабочего стола, а также непосредственно системой тестирования.

\section{Временные характеристики}

Для сравнения возьмем массивы размерностью [10, 100, 200, 300,\dots, 500, 1000]. 
Результаты замеров по результатам экспериментов приведены в Таблице \ref{tbl:time}.

\begin{table}[ht]
	\small
	\begin{center}
		\caption{Замер времени для массивов, размером от 10 до 1000 элементов}
		\label{tbl:time}
		\begin{tabular}{|c|c|c|c|}
			\hline
			& \multicolumn{3}{c|}{\bfseries Время, нс} \\ \cline{1-4}
			\bfseries Длина (символ) & \bfseries Пузырьком & \bfseries Вставками & \bfseries Выбором
			\csvreader{inc/csv/time_sorted.csv}{}
			{\\\hline \csvcoli&\csvcolii&\csvcoliii&\csvcoliv}
			\\\hline
		\end{tabular}
	\end{center}
\end{table}

Отдельно сравним итеративные алгоритмы поиска расстояний Левенштейна и Дамерау--Левенштейна. Сравнение будет производится на основе данных, представленных в Таблице \ref{tbl:time}. Результат можно увидеть на Рисунке \ref{plt:time_sorted_cmp}.

\clearpage

\begin{figure}[ht!]
	\centering
	\begin{tikzpicture}
		\begin{axis}[
			axis lines=left,
			xlabel={Длина массива, эл},
			ylabel={Время, мс},
			legend pos=north west,
			ymajorgrids=true
		]
			\addplot table[x=len,y=Bubble,col sep=comma] {inc/csv/time_sorted.csv};
			\addplot table[x=len,y=Insertion,col sep=comma] {inc/csv/time_sorted.csv};
			\addplot table[x=len,y=Selection,col sep=comma] {inc/csv/time_sorted.csv};
			\legend{Пузырьком, Вставками, Выбором}
		\end{axis}
	\end{tikzpicture}
	\captionsetup{justification=centering}
	\caption{Сравнение времени работы алгоритмов на отсортированных входных данных.}
	\label{plt:time_sorted_cmp}
\end{figure}

В лучшем случае, как и ожидалось, сортировка вставками является наиболее эффективной по времени. При этом, сортировка выбором является наименее эффективной, так так использует дополнительные операции, по сравнению с пузырьковой сортировкой.

\begin{figure}[ht]
	\centering
	\begin{tikzpicture}
		\begin{axis}[
			axis lines=left,
			xlabel={Длина массива, эл},
			ylabel={Время, мс},
			legend pos=north west,
			ymajorgrids=true
		]
			\addplot table[x=len,y=Bubble,col sep=comma] {inc/csv/time_reversed.csv};
			\addplot table[x=len,y=Insertion,col sep=comma] {inc/csv/time_reversed.csv};
			\addplot table[x=len,y=Selection,col sep=comma] {inc/csv/time_reversed.csv};
			\legend{Пузырьком, Вставками, Выбором}
		\end{axis}
	\end{tikzpicture}
	\captionsetup{justification=centering}
	\caption{Сравнение времени работы алгоритмов на отсортированных по убыванию входных данных.}
	\label{plt:time_reversed_cmp}
\end{figure}

В худшем случае сортировка вставками остается наиболее эффективной по времени из рассмотренных. При этом, сортировка пузырьком показывает результат близкий к сортировке выбором.

\begin{figure}[ht]
	\centering
	\begin{tikzpicture}
		\begin{axis}[
			axis lines=left,
			xlabel={Длина массива, эл},
			ylabel={Время, мс},
			legend pos=north west,
			ymajorgrids=true
		]
			\addplot table[x=len,y=Bubble,col sep=comma] {inc/csv/time_random.csv};
			\addplot table[x=len,y=Insertion,col sep=comma] {inc/csv/time_random.csv};
			\addplot table[x=len,y=Selection,col sep=comma] {inc/csv/time_random.csv};
			\legend{Пузырьком, Вставками, Выбором}
		\end{axis}
	\end{tikzpicture}
	\captionsetup{justification=centering}
	\caption{Сравнение времени работы алгоритмов на случайно упорядоченных входных данных.}
	\label{plt:time_random_cmp}
\end{figure}

В общем случае сортировка вставками остается наиболее эффективной по времени из рассмотренных. При этом, сортировка пузырьком оказывается менее эффективной, так как в данном алгоритме элементы массива рассматриваются попарно и на каждой итерации происходит вплоть до $N$ обменов элементов.

Из данных, приведенных в  Таблице \ref{tbl:time}, видно, что сортировка вставками является наиболее эффективной по времени из рассмотренных в любом из трех случаев. При этом, сортировка пузырьком является практически эквивалентной сортировке выбором при входных данных, отсортированных в обратном порядке, но оказывается наиболее затратной при случайном заполнении элементов в массиве.


\section{Вывод}

В данном разделе было произведено сравнение количества затраченного времени вышеизложенных алгоритмов.

Исходя из полученных результатов, можно сделать вывод, что сортировка пузырьком всегда оказывается самой времязатраной, сортировка вставками выигрывает по времени во всех трех случаях из трех представленных, а сортировка выбором оказывается эффективнее пузырьковой сортировки лишь на случайных данных.
