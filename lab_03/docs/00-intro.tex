\chapter*{Введение}
\addcontentsline{toc}{chapter}{Введение}

В данной лабораторной работе будeт рассмотрены различные методы сортировок.

Алгоритмы сортировки имеют широкое практическое применение. Сортировки используются в большом спектре задач, включая обработку коммерческих, сейсмических, космических данных. Часто сортировка является просто вспомогательной операцией для упорядочивания данных, упрощения последующих алгебраических действий над данными.

Сортировка применяется во многих областях программирования, например, базы данных или математические программы. Упорядоченные объекты содержатся в телефонных книгах, ведомостях налогов, в библиотеках, в оглавлениях, в словарях.

В некоторых вычислительных системах на сортировку уходит больше половины машинного времени. Исходя из этого, можно заключить, что либо сортировка имеет много важных применений, либо ею часто пользуются без нужды, либо применяются в основном неэффективные алгоритмы сортировки.

В настоящее время, в связи с экспоненциально возросшими объемами данных, вопрос эффективной сортировки данных снова стал актуальным.

В настоящее время в сети Интернет можно найти результаты производительности алгоритмов сортировки для ряда ведущих центров данных. При этом используются различные критерии оценки эффективности.

Целью данной работы является реализация и изучение следующих алгоритмов:
\begin{itemize}
	\item сортировка пузырьком;
	\item сортировка вставками;
	\item сортировка выбором.
\end{itemize}

Для достижения данной цели необходимо решить следующие задачи:

\begin{itemize}
	\item рассмотреть и изучить сортировки пузырьком, вставками и выбором;
	\item реализовать каждую из этих сортировок;
	\item теоретически оцениьь их трудоемкость;
	\item сравнить их временные характеристики экспериментально.
\end{itemize}
