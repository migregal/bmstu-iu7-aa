\chapter*{Введение}
\addcontentsline{toc}{chapter}{Введение}

В данной лабораторной работе будет рассмотрено расстояние Левенштейна.

Расстояние Левенштейна \cite{Levenshtein} (редакционное расстояние, дистанция редактирования) — метрика, измеряющая разность между двумя последовательностями символов. Она определяется как минимальное количество односимвольных операций (а именно вставки, удаления, замены), необходимых для превращения одной последовательности символов в другую. В общем случае, операциям, используемым в этом преобразовании, можно назначить разные цены.

Впервые задачу поставил в 1965 году советский математик Владимир Левенштейн при изучении последовательностей 0--1, впоследствии более общую задачу для произвольного алфавита связали с его именем.

Расстояние Левенштейна и его обобщения активно применяются: 
\begin{enumerate}[label={\arabic*)}]
	\item для исправления ошибок в слове (в поисковых системах, базах данных, при вводе текста, при автоматическом распознавании отсканированного текста или речи);
	\item в биоинформатике для сравнения генов, хромосом и белков.
	\item для сравнения текстовых файлов утилитой \texttt{diff} и ей подобными (здесь роль «символов» играют строки, а роль «строк» — файлы);
\end{enumerate}

Расстояние Дамерау — Левенштейна (названо в честь учёных Фредерика Дамерау и Владимира Левенштейна) — это мера разницы двух строк символов, определяемая как минимальное количество операций вставки, удаления, замены и транспозиции (перестановки двух соседних символов), необходимых для перевода одной строки в другую. Является модификацией расстояния Левенштейна: к операциям вставки, удаления и замены символов, определённых в расстоянии Левенштейна добавлена операция транспозиции (перестановки) символов.

Целью данной работы является реализация и изучение следующих алгоритмов:
\begin{itemize}
	\item Рекурсивный алгоритм нахождения расстояния Левенштейна.
	\item Рекурсивный алгоритм нахождения расстояния Дамерау--Левенштейна.
	\item Итеративный алгоритм нахождения расстояния Левенштейна с кэшированием.
	\item Итеративный алгоритм нахождения расстояния Дамерау--Левенштейна с \newline кэшированием.
\end{itemize}

Для достижения данной цели необходимо решить следующие задачи:

\begin{enumerate}
	\item изучить расстояния Левенштейна и Дамерау--Левенштейна;
	\item реализовать алгоритмы поиска расстояний: нерекурсивный (с кэшем) и рекурсивные алгоритмы поиска расстояния Левенштейна и расстояния Дамерау--Левенштейна;
	\item замерить процессорное время работы реализаций алгоритмов поиска расстояний;
	\item сравнить временные характеристики, а также затраченную память;
\end{enumerate}
