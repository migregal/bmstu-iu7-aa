\chapter{Экспериментальная часть}

В данном разделе будет проведено функциональное тестирование разработанного программного обеспечения. Так же будет произведено измерение временных характеристик и характеристик по памяти каждого из реализованных алгоритмов. 

Для проведения подобных экспериментов на языке программирования \texttt{Golang} \cite{golang}, используется специальный пакет \texttt{testing} \cite{gotest}. Данный пакет предоставляет инструменты для измерения процессорного времени и объема памяти, использованных конкретным алгоритмом в ходе проведения 
эксперимента.

\section{Тестирование}

В таблице \ref{tabular:functional_test} приведены функциональные тесты для алгоритмов вычисления расстояния Левенштейна и Дамерау--Левенштейна.

\begin{table}[h]
	\begin{center}
		\caption{\label{tabular:functional_test} Функциональные тесты}
		\begin{tabular}{|c|c|c|c|}
			\hline
            & & \multicolumn{2}{c|}{\bfseries Ожидаемый результат}    \\ \cline{3-4}
			\bfseries Строка 1  & \bfseries Строка 2 & \bfseries Левенштейн & \bfseries Дамерау — Левенштейн
			\csvreader{inc/csv/functional-test.csv}{}
			{\\\hline \csvcoli&\csvcolii&\csvcoliii&\csvcoliv}
			\\\hline
		\end{tabular}
	\end{center}
\end{table}

При проведении функционального тестирования, полученные результаты работы программы совпали с ожидаемыми. Таким образом, функциональное тестирование пройдено успешно.

\section{Технические характеристики}

Технические характеристики устройства, на котором выполнялось исследование:

\begin{itemize}
	\item Процессор: Intel Core™ i5-8250U \cite{i5} CPU @ 1.60GHz.
	\item Память: 32 GiB.
	\item Операционная система: Ubuntu \cite{ubuntu} Linux \cite{linux} 20.04 64-bit.

\end{itemize}

Исследование проводилось на ноутбуке, включенном в сеть электропитания. Во время тестирования ноутбук был нагружен только встроенными приложениями окружения рабочего стола, окружением рабочего стола, а также непосредственно системой тестирования.

\section{Временные характеристики}

Результаты замеров по результатам экспериментов приведены в Таблице \ref{tbl:time}. В данной таблице для значений, для которых тестирование не выполнялось, в поле результата находится "\ - ".

\begin{table}[ht]
	\small
	\begin{center}
		\caption{Замер времени для строк, размером от 5 до 200}
		\label{tbl:time}
		\begin{tabular}{|c|c|c|c|c|}
			\hline
			& \multicolumn{4}{c|}{\bfseries Время, нс} \\ \cline{2-5}
			& \multicolumn{2}{c|}{\bfseries Левенштейн} & \multicolumn{2}{c|}{\bfseries Дамерау -- Левенштейн} \\ \cline{2-5}                  
			\bfseries Длина (символ) & \bfseries Рекурсивный & \bfseries Итеративный & \bfseries Рекурсивный & \bfseries Итеративный
			\csvreader{inc/csv/time.csv}{}
			{\\\hline \csvcoli&\csvcolii&\csvcoliii&\csvcoliv&\csvcolv}
			\\\hline
		\end{tabular}
	\end{center}
\end{table}

Отдельно сравним итеративные алгоритмы поиска расстояний Левенштейна и Дамерау--Левенштейна. Сравнение будет производится на основе данных, представленных в Таблице \ref{tbl:time}. Результат можно увидеть на Рисунке \ref{plt:time_levenshtein}.

\begin{figure}[ht]
	\centering
	\begin{tikzpicture}
		\begin{axis}[
			axis lines=left,
			xlabel=Длина строк,
			ylabel={Время, нс},
			legend pos=north west,
			ymajorgrids=true
		]
			\addplot table[x=len,y=Levenstein,col sep=comma] {inc/csv/cmp_time_it_it.csv};
			\addplot table[x=len,y=DLevenshtein,col sep=comma] {inc/csv/cmp_time_it_it.csv};
			\legend{Левенштейн, Дамерау -- Левенштейн}
		\end{axis}
	\end{tikzpicture}
	\captionsetup{justification=centering}
	\caption{Сравнение времени работы алгоритма поиска расстояния Левенштейна и Дамерау -- Левенштейна}
	\label{plt:time_levenshtein}
\end{figure}

При длинах строк менее 30 символов разница по времени  между итеративными реализациями незначительна, однако при увеличении длины строки алгоритм поиска расстояния Левенштейна оказывается быстрее вплоть до полутора раз (при длинах строк равных 200). Это обосновывается тем, что у алгоритма поиска расстояния Дамерау-Левенштейна задействуется дополнительная операция, которая замедляет алгоритм.\newline


Так же сравним рекурсивную и итеративную реализации алгоритма поиска расстояния Левенштейна. Данные представлены в Таблице \ref{tbl:time} и отображены на Рисунке \ref{plt:time_levenshtein_cmp}.

\begin{figure}[ht]
	\centering
	\begin{tikzpicture}
		\begin{axis}[
			axis lines=left,
			xlabel=Длина строк,
			ylabel={Время, мс},
			legend pos=north west,
			ymajorgrids=true
		]
			\addplot table[x=len,y=Levenshtein,col sep=comma] {inc/csv/cmp_time_it_rec.csv};
			\addplot table[x=len,y=ItLevenshtein,col sep=comma] {inc/csv/cmp_time_it_rec.csv};
			\legend{Левенштейн (Рекурсивная), Левенштейн}
		\end{axis}
	\end{tikzpicture}
	\captionsetup{justification=centering}
	\caption{Сравнение времени работы рекурсивной и итеративной кэширующей реализаций алгоритма Левенштейна.}
	\label{plt:time_levenshtein_cmp}
\end{figure}

На Рисунке \ref{plt:time_levenshtein_cmp} продемонстрировано, что рекурсивный алгоритм становится менее эффективным (вплоть до 56 раз при длине строк равной 5 элементов), чем итеративный.

Из этого можно сделать вывод о том, что при малых длинах строк (1--2 символа) предпочтительнее использовать рекурсивные алгоритмы, однако при обработке более длинных строк (болеее 3 символов) итеративные алгоритмы оказываются многократно более эффективными и рекомендованы к использованию.\newline

Из данных, приведенных в  Таблице \ref{tbl:time}, видно, что итеративные алгоритмы становятся более эффективными по времени при увеличении длин строк, работая приблизительно в 308 млн. раз (Левенштейн) и 203 млн. раз (Дамерау -- Левенштейн) быстрее, чем рекурсивные (при длинах строк равных 200). Однако, при малых длинах (1 -- 2 элемента) рекурсивные алгоритмы являются более эффективными (вплоть до 3 раз), чем итеративные.

Кроме того, согласно данным, приведенным в Таблице \ref{tbl:time}, рекурсивные алгоритмы при длинах строк более 15 элементов не пригодны к использованию в силу экспоненциально роста затрат процессорного времени, в то время, как затраты итеративных алгоритмов по времени линейны. 

\section{Характеристики по памяти}

Результаты замеров по результатам экспериментов приведены в Таблице \ref{tbl:memory}. В данной таблице для значений, для которых тестирование не выполнялось, в поле результата находится "\ - ".

\begin{table}[ht]
	\small
	\begin{center}
		\caption{Замер расхода памяти для строк, размером от 5 до 200}
		\begin{tabular}{|c|c|c|c|c|}
			\hline
			& \multicolumn{4}{c|}{\bfseries Память, байт} \\ \cline{2-5}
			& \multicolumn{2}{c|}{\bfseries Левенштейн} & \multicolumn{2}{c|}{\bfseries Дамерау -- Левенштейн} \\ \cline{2-5}                  
			\bfseries Длина (символ) & \bfseries Рекурсивный & \bfseries Итеративный & \bfseries Рекурсивный & \bfseries Итеративный
			\csvreader{inc/csv/memory.csv}{}
			{\\\hline \csvcoli&\csvcolii&\csvcoliii&\csvcoliv&\csvcolv}
			\\\hline
		\end{tabular}
		\label{tbl:memory}
	\end{center}
\end{table}

Из данных, приведенных в Таблице \ref{tbl:memory}, видно, что рекурсивные алгоритмы являются более эффективными по памяти, так как не используют дополнительной памяти в своей работе, в то время как итеративные алгоритмы затрачивают память линейно пропорционально длинам обрабатываемых строк.

В связи с этим, при недостаточном объеме памяти, рекомендуются использовать рекурсивные алгоритмы, так как они не используют дополнительной памяти в процессе работы.

Кроме того, итеративный алгоритм Дамерау -- Левенштейна расходует больше памяти, что связано с хранением третьей строки "кэша" для учета допролнительной операции.

\section{Сравнительный анализ алгоритмов}

Приведенные характеристики показывают нам, что рекурсивная реализация алгоритма очень сильно проигрывает по времени. В связи с этим, рекурсивные алгоритмы следует использовать лишь для малых размерностей строк (1-2 символа) или при малом объеме оперативной памяти.

Так как во время печати очень часто возникают ошибки связанные с транспозицией букв, алгоритм поиска расстояния Дамерау -- Левенштейна является наиболее предпочтительным, не смотря на то, что он проигрывает по времени и памяти алгоритму Левенштейна. 

По аналогии с первым абзацем можно сделать вывод о том, что рекуррентный алгоритм поиска расстояния Дамерау-Левенштейна будет более затратным по времени по сравнению с итеративной реализацией алгоритма поиска расстояния Дамерау --Левенштейна с кешированием.

\section{Вывод}

В данном разделе было произведено сравнение количества затраченного времени и памяти вышеизложенных алгоритмов. Наименее затратным по времени оказался рекурсивный алгоритм нахождения расстояния Дамерау -- Левенштейна.

Для обработок малых длин строк (1 -- 2 символа) предпочтительнее использовать рекурсивные алгоритмы, в то время как для остальных случаев рекомендуются использовать итеративные реализации. Однако, стоит учитывать дополнительные затраты по памяти, возникающие при использовании итеративных алгоритмов.
