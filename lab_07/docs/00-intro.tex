\chapter*{Введение}
\addcontentsline{toc}{chapter}{Введение}

Словарь или ассоциативный массив -- абстрактный тип данных (интерфейс к хранилищу данных, позволяющий хранить пары вида (ключ; значение) и поддерживающий операции добавления пары, а также поиска и удаления пары по ключу.

В паре $(k,v)$ значение $v$ называется значением ассоциированным с ключом $k$. Семантика и названия вышеупомянутых операций в разных реализациях ассоциативного массива могут отличаться.

Ассоциативный массив с точки зрения интерфейса удобно рассматривать как обычный массив: в котором в качестве индексов можно использовать не только целые числа, но и значения других типов -- например, строк.

Поддержка ассоциативных массивов есть во многих языках программирования высокого уровня, таких, как \texttt{Perl}, \texttt{PHP}, \texttt{Python}, \texttt{JavaScript} и других. Для языков, не имеющих встроенных средств для работы с ассоциативными массивами, существует множество реализаций в виде библиотек.

Целью данной лабораторной работы является изучение способа эффективного оп времени и памяти поиска по словарю. Для достижения данной цели необходимо решить следующие задачи:

\begin{itemize}
	\item исследовать основные алгоритмы поиска по словарю;
	\item привести схемы рассматриваемых алгоритмов;
	\item описать использующиеся структуры данных;
	\item описать структуру разрабатываемого программного обеспечения;
	\item определить средства программной реализации;
	\item определить требования к программному обеспечению;
	\item привести сведения о модулях программы;
	\item провести тестирование реализованного программного обеспечения;
	\item провести экспериментальные замеры временных характеристик реализованных алгоритмов.
\end{itemize}
