\chapter{Конструкторская часть}

В данном разделе представлены схемы рассматриваемых алгоритмов.

\section{Схемы}

\imgw{brute_scheme}{ht!}{0.7\textwidth}{Схема алгоритма поиска с полным перебором}

\imgw{bin_scheme}{ht!}{\textwidth}{Схема алгоритма с бинарным поиском}
\clearpage

\imgw{freq_scheme}{ht!}{0.8\textwidth}{Схема алгоритма с частотным анализом}
\clearpage

\section{Описание структуры программного обеспечения}

На Рисунке \ref{img:uml} представлена \texttt{uml}--диаграмма  разрабатываемого программного обеспечения.

\imgw{uml}{ht!}{\textwidth}{Структура программного обеспечения}

\section{Описание структур данных}

Для реализации конвейерных вычислений, введем некоторые пользовательские типы данных:

\begin{itemize}
    \item \texttt{Dict} -- тип данных, описывающий ассоциативный массив с ключами типа \texttt{string};
    \item \texttt{DictArray} -- тип данных, описывающий массив ассоциативных массивов;
    \item \texttt{Freq} -- структура, описывающая сегмент частотного анализа ассоциативного массива.
    \item \texttt{FreqArray} -- тип данных, описывающий результат частотного анализа ассоциативного массива.
\end{itemize}

Рассмотрим каждый из введенных пользовательских данных.
\clearpage

\subsection{Описание пользовательского типа данных \texttt{Freq}}

\listingfile{types.go}{types_3}{Go}{Описание пользовательского типа данных \texttt{Freq}}{linerange={7-11}}

Здесь:
\begin{itemize}
    \item \texttt{l} -- строка, содержащая признак сегмента -- букву из используемого алфавита;
    \item \texttt{cnt} -- частотная характеристика для данного сегмента;
    \item \texttt{darr} -- ассоциативный массив, содержащий записи, соответствующие данному сегменту.
\end{itemize}

\section{Вывод}

На основе теоретических данных, полученных из аналитического раздела, были построены схемы реализуемыъ алгоритмов поиска в ассоциативных массивах (Рисунки \ref{img:brute_scheme}--\ref{img:freq_scheme}).

Так же, было приведено описание пользовательских типов данных, вводимых в рамках реализации конвейерных вычислений.