\chapter{Конструкторская часть}

В данном разделе будут рассмотрены схемы алгоритмов умножения матриц и модель вычислений.

\section{Схемы}
На Рисунке \ref{img:conveyer} представлена схема организации конвейерных вычислений на примере конвейера с тремя лентами. 

\imgw{conveyer}{ht!}{0.7\textwidth}{Схема организации конвейерных вычислений}

Заметим, так же, что вычисление хеш-суммы файла не зависит от ранее вычисленных значений. В связи с этим, можно реализовать многопоточное вычисление хеш-сумм файлов, посредством так называемых \texttt{fan-out} и \texttt{fan-in} подходов.

Таким образом, схема \ref{img:conveyer} может быть преобразована следующим образом:

\imgw{conveyer_parallel}{ht!}{0.7\textwidth}{Схема организации конвейерных вычислений}

Отдельно стоит отметить, что в алгоритме, рассматриваемом в рамках данной лабораторной работы, отсутствует первая очередь конвейера, т.к. первая лента и является основным поставщиком задач для обработки.

\section{Описание структуры программного обеспечения}

На Рисунке \ref{img:idef0} представлена \texttt{idef0}--диаграмма работы описанного выше конвейера.

\imgw{idef0}{ht!}{\textwidth}{Idef0--диаграмма работы конвейера}

\section{Описание структур данных}

Для реализации конвейерных вычислений, введем некоторые пользовательские типы данных:

\begin{itemize}
    \item \texttt{fwOutput} -- структура, описывающая результат обработки задачи первой лентой конвейера;
    \item \texttt{dOutput} -- структура, описывающая результат обработки задачи второй лентой конвейера;
    \item \texttt{Output} -- структура, описывающая результат обработки задачи конвейером.
\end{itemize}

Рассмотрим каждый из введенных пользовательских данных.
\clearpage

\captionsetup{justification = raggedright, singlelinecheck = false}

\subsection{Описание пользовательского типа данных fwOutput}

\listingfile{types.go}{fw_output}{Go}{Определение пользовательских типов данных. Часть 1}{linerange={8-14}}

Здесь:
\begin{itemize}
    \item \texttt{path} -- строка, содержащая путь до рассматриваемого в задаче файла;
    \item \texttt{processTime} -- длительность обработки задачи первой лентой конвейера;
    \item \texttt{queueStart} -- момент попадания задачи в очередь второй ленты.
\end{itemize}

\subsection{Описание пользовательского типа данных dOutput}

\listingfile{types.go}{d_output}{Go}{Определение пользовательских типов данных. Часть 2}{linerange={15-25}}

Здесь:
\begin{itemize}
    \item \texttt{path} -- строка, содержащая путь до рассматриваемого в задаче файла;
    \item \texttt{sum} -- \texttt{md5}--хеш содержимого файла, обрабатываемого в рамках задачи;
    \item \texttt{err} -- поле, содержащее, при наличии таковой, описание ошибки обработки задачи;
    \item \texttt{filewalker} --  длительность обработки задачи первой лентой конвейере;
    \item \texttt{queue} -- длительность ожидания в очереди второй ленты конвейера;
    \item \texttt{processTime} -- длительность обработки задачи второй лентой конвейера;
    \item \texttt{queueStart} -- момент попадания задачи в очередь третьей ленты.
\end{itemize}

\subsection{Описание пользовательского типа данных Output}

\listingfile{types.go}{output}{Go}{Определение пользовательских типов данных. Часть 3}{linerange={26-35}}

Здесь:
\begin{itemize}
    \item \texttt{Path} -- строка, содержащая путь до рассматриваемого в задаче файла;
    \item \texttt{Sum} -- \texttt{md5}--хеш содержимого файла, обрабатываемого в рамках задачи;
    \item \texttt{Filewalker} --  длительность обработки задачи первой лентой конвейере;
    \item \texttt{DigesterQueue} -- длительность ожидания в очереди второй ленты конвейера;
    \item \texttt{Digester} -- длительность обработки задачи второй лентой конвейера;
    \item \texttt{Queue} -- lkbntkmyjcnmдлительность ожидания в очереди 3ей ленты.
\end{itemize}

\section{Вывод}

На основе теоретических данных, полученных из аналитического раздела, была построена схема организации конвейерных вычислений на примере конвейере с тремя лентами (Рисунок \ref{d:conveyer}).

Так же, было приведено описание пользовательских типов данных, вводимых в рамках реализации конвейерных вычислений (Листинги \ref{lst:fw_output} -- \ref{lst:output}).