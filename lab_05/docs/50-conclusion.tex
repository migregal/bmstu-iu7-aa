\chapter*{Заключение}
\addcontentsline{toc}{chapter}{Заключение}

В данной лабораторной работе была конвейерная организация вычислений. 

Среди рассмотренных алгоритмов наиболе эффективным по времени яляется параллельный алгоритм умножения матриц по строкам, так как в нем отсутствуют лишние обращения к памяти. 

В связи с вышуказанным, параллельный алгоритм умножения по строкам является предпочтительным при обработке больших матриц в многопоточном окружении, однако, при работе с матрицами малых размерностей (меньше 64), стандартный алгоритм умножения становится более эффективным в связи с дополнительными затратами на организацию параллельности вычислений (создание потоков, организация совместного доступа к ресурсам).

В рамках выполнения работы решены следующие задачи.
\begin{itemize}
    \item исследованы основы конвейерных вычислений; 
	\item исследованы основные методы организации конвейерных вычислений;
	\item проведено сравнение существующих методов организации конвейерных вычислений;
	\item приведены схемы рассматриваемых алгоритмов, а именно:
	\begin{itemize}
	    \item схема конвейера, содержащего 3 ленты;
	    \item схема конвейера, содержащего 3 ленты и реализующего \texttt{fan--in}--\texttt{fan--out} подходы;
	\end{itemize}
	\item описаны использующиеся структуры данных;
	\item описана структура разрабатываемого програмного обеспечения;
	\item определены средства программной реализации;
	\item определены требования к программному обеспечению;
	\item приведены сведения о модулях программы;
	\item проведено тестирование реализованного программного обеспечения;
	\item проведены экспериментальные замеры временных характеристик реализованного конвейера.
\end{itemize}