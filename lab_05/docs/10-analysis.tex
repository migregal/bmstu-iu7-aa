\chapter{Аналитическая часть}

В данном разделе рассматривается понятие конвейерной обработки. 

Так же, рассматривается предметная область лабораторной работы:
\begin{itemize}
    \item описание решаемой задачи;
    \item описание каждой из выделенных стадий реализуемого алгоритма решения данной задачи.
\end{itemize}

\section{Конвейерная обработка}

В рамках данной лабораторной работы конвейерной обработкой будет считать так называемую "синхронную"\ конвейерную обработку \cite{kormen}. В конвейерах данного типа отсутствуют очереди между различными "лентами" конвейера. В связи с этим, на конвейерах, организованных подобным образом, часть лент может оказаться заблокированной, т.к. для продолжения выполнения кода необходимо дождаться освобождения последующей ленты конвейера. 

Тем не менее, теоритически подобный подход является менее затратным по памяти, т.к. не использует дополнительных структур для реализации очередей.

\section{Предметная область}

В качестве алгоритма, реализованного для распределения на конвейере, было выбрано вычисление кеш-сумм каждого из файлов в заданном каталоге файловой системы. Данный алгоритм состоит из 3-х этапов:
\begin{itemize}
\item получение полного пути файла;
\item вычисление кеш-суммы (алгоритм \texttt{md5} \cite{md5}) содержимого файла;
\item агрегация полученных результатов и сохранение в результирующую структуру.
\end{itemize}

\section{Вывод}

В данной работе стоит задача реализации асинхронных конвейерных вычислений. Были рассмотрены особенности построения конвейерных вычислений.
\clearpage

Входными данными для программного обеспечения служит путь до файла или директории в файловой системе.

Выходными данными явлются:
\begin{itemize}
    \item упорядоченный по имени список обработанных файлой и соответствующие им хеш--суммы;
    \item усредненные временные характеристики конвейера: среднее время обработки заявки каждой из лент, а так же - усредненное время ожидания в каждой из очередей конвейера.
\end{itemize}

На программное обеспечения накладываются следующие ограничения:
\begin{itemize}
    \item входные данные должны быть корреткными т.е.:
    \begin{itemize}
        \item введенный путь должен быть корреткным;
        \item введенному пути должен существовать файл/директория
    \end{itemize}
    \item программа должна иметь соответствующие права времени выполнения для чтения файлов.
\end{itemize}